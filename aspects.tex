\section{On ERCI and its Pareto Front}\label{sec:convex}
\subsection{ERCI as a multi-objective problem}
As may already be apparent to the reader, there is a natural trade off
between probability of generating paths in $\varphi$ (from her
onwards: \emph{the performance}) and causal entropy induced by a
policy (from here onwards: \emph{the randomization}).  In particular,
we are interested in understanding the combinations of $\scthreshold$
and $\randomness$ that allow to solve the (core) ERCI problem. To this
end, we cast ERCI as a variant of multi-objective optimization and
studying its Pareto Front.

% It is convenient to consider this geometrically.
To begin, given a fixed ERCI instance, observe that a scheduler $\sched$
induces a point $x_\sched$:
\begin{equation}
  x_\sched \eqdef \Big\langle \Pr(X_\varphi \mid \sched), H(\sched) \Big\rangle \in [0,1] \times [0,\infty).  
\end{equation}

To ease notation, for $x_\sched = \langle \scp,\rndp \rangle$ we use
$\scp_\sched \eqdef \scp$ and $\rndp_\sched \eqdef \rndp$.

For a $\pOne$-policy $\pOneSched$, we say that $\pOneSched$
\emph{guarantees} a point $x_\pOne \eqdef \langle \scp, \rndp
\rangle$, if for every policy $\pTwoSched$, using $\sched = \langle
\pOneSched, \pTwoSched \rangle$, we have $\scp_\sched \geq \scp$ and
$\rndp_\sched \geq \rndp$. Thus, a point is guaranteed if no matter
what policy $\pTwo$ uses, $x_\sched$ will induce a point no worse
w.r.t.\ to either randomization or performance.

We then define the following elementary sets: For any $\pOneSched$,
that $\pOneSched$ guarantees $\langle \scp, \rndp \rangle$, we define
the set of guaranteed points:
\begin{equation}\label{eq:guaranteed}
\solutions[\pOneSched] = \{ \langle \scp', \rndp' \rangle \mid \scp' \leq \scp \wedge \rndp \leq \rndp'\},
\end{equation}

Any point that can be guaranteed by some $\pOneSched$ is called
\emph{achievable}, i.e., the achievable points are: $ \solutions =
\bigcup_{\pOneSched} \solutions[\pOneSched]$.  Importantly, it holds
that the ERCI problem is satisfiable iff $\langle \scthreshold,
\randomness \rangle$ is achievable.  \sj{We should clarify satisfiable
lingo} Thus, to solve ERCI instances, we start by characterizing
$\solutions$.

\begin{example}
	
\end{example}

\begin{proposition}
  The set $\solutions$ is convex. 
\end{proposition}
\begin{proof}[Proof Sketch]
  Recall that a set is convex, if it is closed under
  convex-combinations\footnotemark. Consider two points
  $\langle \scthreshold, \randomness \rangle, \langle \scthreshold',
  \randomness' \rangle \in \solutions$ achieved by $\pOneSched$ and
  $\pOneSchedPrime$ resp. Consider the new policy,
  $\pOneSchedPrimePrime$ defined by using $\pOneSched$ with
  probability $q$ and $\pOneSchedPrime$ with probability $1 - q$.
  By construction, this new policy as performance at least $q\cdot \scthreshold + (1 - q)\cdot \scthreshold'$.
  Similarly, via chain rule, the causal entropy of this new policy is:
  \begin{equation}
    \begin{split}
      H_\tau(\sigma) &\geq q \cdot H( \rv{A}^{\pOne}_{1:\tau'} \mid\mid \rv{S}_{1:\tau} \mid \pOneSched) \\
      &~~+ ~(1 - q)  \cdot H( \rv{A}^{\pOne}_{1:\tau'} \mid\mid \rv{S}_{1:\tau} \mid \pOneSchedPrime)\\
      &\geq~q\cdot\randomness + (1 - q)\cdot \randomness'.
    \end{split}
  \end{equation}
  Finally, observe that the convex combination of policies is also a
  policy\hfill $\blacksquare$.
  % Let $\sched'$ and $\sched''$ be two schedulers with induced
  % solution $\langle \scthreshold', \randomness' \rangle$ and
  % $\langle \scthreshold'', \randomness'' \rangle$
\end{proof}
\footnotetext{That is, $y, y' \in Y$ implies for
    every $w \in [0,1]$ that $w \cdot y + (1-w) \cdot y \in Y$}


We can characterize the solutions using the Pareto-front. Therefore, we say that a point $x = \langle \scp,\rndp \rangle$ is dominated by $x' = \langle \scp', \rndp' \rangle$, in symbols $x \prec x'$, if $x'$ is strictly larger in at least one dimension and not smaller in the other, that is, 
\begin{equation}
  \langle \scp', \rndp' \rangle 	 \prec \langle \scp', \rndp' \rangle \eqdef \Big( \scp \leq \scp' \land \rndp \leq \rndp' \land \big(\scp \neq \scp' \lor \rndp \leq \rndp'\big)	\Big)
\end{equation}
The Pareto-front $\pareto{\solutions}$ of $\solutions$ is given as \[ \pareto{\solutions} \eqdef \{ x \in \solutions \mid \forall x' \in \solutions, x \not\prec x'  \}. \]
Now, it holds that the ERCI problem is satisfiable iff there exists a  $x \in \pareto{\solutions}$ such that $\langle \scthreshold, \randomness \rangle \prec x$.  

The important aspect of this Pareto-front is that it gives a natural approximation scheme in the following sense.
For any subset $\pareto{} \subseteq \pareto{\solutions}$, if there exists an  $x \in \pareto{}$ such that $\langle \scthreshold, \randomness \rangle \prec x$, then the ERCI Problem must be satisfiable, 
and if there exists an $x \in \pareto{}$ such that $x \prec \langle \scthreshold, \randomness \rangle$ then the ERCI problem is not satisfiable. 

\begin{example}
	
\end{example}

We can improve upon the construction of the Pareto-front by exploiting its convexity.

While $\solutions$ is defined as a subset of $[0,1] \times [0,\infty)$, we can restrict ourselves to the domain in which we can actually trade performance for randomness. 
We define 
$\rndopt \eqdef \max \{ \rndp \mid \exists \scp \text{ s.t. } \langle \scp, \rndp \rangle \in \solutions  \} $, i.e., the largest randomness that can be guaranteed by any $\pOne$-policy. 
Likewise, we define 
$\scopt \eqdef \max \{ \scp \mid \exists \rndp \text{ s.t. } \langle \scp, \rndp \rangle \in \solutions  \} $, i.e., the largest performance that can be guaranteed by any $\pOne$-policy. 
Then, we define 
$\scmin \eqdef \max \{ \scp \mid \langle \scp, \rndopt \rangle  \in \solutions \}$, the best performance that $\pOne$ can guarantee while guaranteeing optimal randomness. 
Likewise, we define  the analogous $\rndmin \eqdef \max \{ \scp \mid \langle \scp, \rndopt \rangle  \in \solutions \}$.
We thus obtain two points on the Pareto-curve: $\langle \scmin, \rndopt \rangle$ and $\langle \scopt, \rndmin \rangle$, and intuitively, we can trade between these two points following the Pareto curve.


Finally, it is helpful to think about the Pareto-curve as a function of either $\rndp$. 
We define a characteristic function $\solfuncp\colon [\rndmin,\rndopt] \rightarrow [\scmin,\scopt]$ such that $\solfuncp(\rndp) = \max \{ \scp \mid \langle \scp, \rndp \rangle \in \solutions \} \cup \{ 0 \}$.  
\begin{proposition}
	$\solfuncp$ is smooth monotonically decreasing. 
\end{proposition}
Monotone decreasing follows directly from convexity and using the adequate domains. 
{\color{red}how do we know smoothness? Isn't that a consequence of the rationality-construction that we only have available later?}
In particular, the set is not a finite polytope, there exist infinitely many vertices.


With these facts, we are now well-equiped to develop the algorithms in Sec.~\ref{sec:mdps} for MDPs and Sec.~\ref{sec:sgs} for SGs.
We use the remainder of this section to motivate the choice for our problem formulation.

\subsection{ERCI versus reactive control improvisation}

Before we continue, we compare ERCI with the reactive, i.e. 2-player control improvisation framework from~\cite{}:
\begin{mdframed}
\textbf{The Deterministic Reactive Control Improvisation (DRCI) Problem}:
Given a (\emph{deterministic}) SG $\sg$, finite path sets $X_\psi$ and $X_\varphi$,  thresholds $\scthreshold \in (0,1)$ and $\ppthreshold \in (0,1)$,  find a $\pOne$-policy $\pOneSched \in \POneScheds$  such that for every $\pTwo$-policy $\pTwoSched \in \PTwoScheds$: 
\begin{compactenum}
\item (\emph{hard constraint}) $\Pr(X_\psi \mid \sched) \geq 1$,
	\item (\emph{soft constraint)} $\Pr(X_\varphi \mid \sched) \geq \scthreshold$,
	\item (\emph{randomness}) $\Pr(\path \mid \sched) \leq \ppthreshold \text{ for all } \path \in \Paths{\sg}$,
\end{compactenum}
where  $\sched = \langle \pOneSched, \pTwoSched \rangle$.
\end{mdframed}
While DRCI is only applied to deterministic SGs in \cite{DBLP:conf/cav/FremontS18}, there is nothing in the definition that prevents its application to the general class of SGs\footnote{However, this does not mean that the algorithm to compute a solution carries over to the general case}.
We observe that then, the only difference between ERCI and DRCI is that we use entropy rather than an upper bound on the probability of a path.  Let us first illustrate why we choose a different type of randomness constraint.
\begin{figure}
\begin{tikzpicture}
	\node[sstate, initial, initial text=] (s0) {$s_0$};
	\node[sstate,right=of s0, yshift=2.5em] (s2) {$s_2$};
	\node[sstate,above=0.4cm of s2] (s1) {$s_1$};
	
	\node[right=of s0,yshift=0.3em,xshift=0.2em] (sdots) {$\vdots$};
	\node[sstate,right=of s0, yshift=-2.5em] (sn) {$s_n$};
	\node[sstate,right=of s1] (t1) {$t_1$};
	\node[right=of t1] (tdots) {$\hdots$};
	\node[sstate,right=of tdots] (tn) {$t_m$};
	\node[sstate, right=2.2cm of s0] (u) {$u$};
	\node[sstate, right=of u, yshift=1em] (v1) {$v_1$};
	\node[sstate, right=of u, yshift=-1em] (v2) {$v_2$};
	\node[right=0.5cm of v1] {$\hdots$};
	\node[right=0.5cm of v2] {$\hdots$};
	
	
	
	\draw[->] (t1) -- node[elab] {$1$} (tdots);
	\draw[->] (tdots) -- node[elab] {$1$} (tn);
	
	\draw[->] (s0) -- node[elab] {$\nicefrac{1}{n}$} (s1);
	\draw[->] (s0) -- node[elab,below] {$\nicefrac{1}{n}$} (s2);
	\draw[->] (s0) -- node[elab,below] {$\nicefrac{1}{n}$} (sn);
	
	\draw[->] (s1) -- node[elab,above] {$1$} (t1);
	\draw[->] (s2) -- node[elab,near end,above] {$1$} (u);
	\draw[->] (sn) -- node[elab,near end,below] {$1$} (u);
	
	\draw[->] (u) -- node[actnode] {} node[elab,above, near start] {$a$}  node[elab,above, near end] {$1$} (v1);
	\draw[->] (u) -- node[actnode] {} node[elab,below, near start] {$b$} node[elab,below, near end] {$1$} (v2);
	
	
\end{tikzpicture}
\caption{}	
\end{figure}

\begin{example}
	Consider the SG (actually, an MDP) in Fig.~\ref{fig:drciOnSgs}. 
	First consider that under each scheduler, the path from $s_0$ to $t_m$ has probability $\nicefrac{1}{n}$. In particular, this means that a feasible DRCI instance (applied to an SG) must have $\ppthreshold \geq \nicefrac{1}{n}$. At the same time, every path in the SG already has probability at most $\nicefrac{1}{n}$, and thus, every schedulder that satisfies the randomness constraint for $\delta = 1$ satisfies it for any $\ppthreshold \geq \nicefrac{1}{n}$. Thus, DRCI does not allow us to enforce any kind of randomization in the $\pOne$-policy. 
\end{example}

On the other hand, for deterministic SGs, all randomization is due to the random behavior of $\pOne$. Roughly speaking, every path is then just good (i.e., ending in a target state) or bad (i.e., ending in a sink state). The randomization criterion then may just ensure that we do not select the same path with too much probability, which intuitively means that the $\pOne$-player must ensure in every step that there are a sufficient number of paths from the next state (no matter what $\pTwo$-player does). 
As long as there exist a sufficient number of paths, randomizing (uniformly) over those paths leads to a solution.

	For deterministic SGs, the set of feasbile ERCI and DRCI instances coincide.
\begin{theorem}
For any deterministic SG with some path sets $X_\varphi$, $X_\psi$ and threshold $\scthreshold$, 
there exists an $\pOne$-policy solving the DRCI problem with threshold $\ppthreshold$ iff there exists a $\pOne$-policy solving the ERCI problem with threshold $\randomness = f(\ppthreshold)$ for an adequate (computable) $f$.
\end{theorem}
We note that the policies solving the two problems do in general not coincide. 

\subsection{Causal Entropy versus Entropy}


%%% Local Variables:
%%% mode: latex
%%% TeX-master: "main"
%%% End:
