\section{ERCI as multi-objective optimization}\label{sec:convex}
There is a natural trade off
between probability of generating paths in $\varphi$ (from here
onwards: \emph{the performance}) and causal entropy induced by a
policy (\emph{the randomization}).  In particular,
we are interested in understanding the combinations of $\scthreshold$
and $\randomness$ that allow to solve the (core) ERCI problem. To this
end, we cast ERCI as an instance of a multi-objective optimization problem, and
study its Pareto front. Some ideas are inspired by variants of multi-objective analysis of MDPs with multiple soft constraints, e.g.~\cite{DBLP:conf/stacs/ChatterjeeMH06,DBLP:conf/tacas/EtessamiKVY07}.


\begin{figure*}
\begin{subfigure}{0.19\textwidth}
\centering
\begin{tikzpicture}[scale=2.5]	
     	
  	\draw (0.85, 0.3) node[cross=2pt,color=blue] (c1) {};
  	\draw (0.78, 0.9) node[cross=2pt,color=blue] (c2) {};
  	\draw (0.6, 0.8) node[cross=2pt,color=blue] (c3) {};
  	
  	\node at (0.6,0.3) (x1) {};
  		\draw[fill=green!10!white,draw=green] (0,0) rectangle (x1);
  		 \draw[thick, ->] (-0.05, 0) -- (1, 0) node[below]{$\rndp$};
  	\draw[thick, ->] (0, -0.05) -- (0, 1) node[above] {$\scp$};

\fill[fill=blue!10] (c1.center)--(c2.center)--(c3.center);
	
	  \fill[fill=blue!20,pattern=north west lines,pattern color=blue] (c1.center)--(c2.center)--(c3.center);
	
  	
  	\draw[dashed] (0,0) |- (c1);
  	\draw[dotted] (0,0) |- (c2);
  	\draw[dotted] (0,0) |- (c3);
  	\draw[dotted] (0,0) -| (c1);
  	\draw[dotted] (0,0) -| (c2);
  	\draw[dashed] (0,0) -| (c3);
  	
  	 
  	\draw (x1) node[circle,fill=green, inner sep=1.5pt,draw=green!50!black] {};
  
  	
  	
\end{tikzpicture}
\caption{Guaranteed points $\solutions_\pOneSched$}
\label{fig:geom:guarantee}
\end{subfigure}
\begin{subfigure}{0.19\textwidth}
\centering
\begin{tikzpicture}[scale=2.5]	
     	
  	\node at (0.6,0.3) (x1) {};
  		\node at (0.3,0.8) (x2) {};
  	
  		\node at (0.5,0.57778) (x3) {};
  	
  		\draw[fill=green!10!white,draw=green] (0,0) rectangle (x1);
  			\draw[fill=green!10!white,draw=green] (0,0) rectangle (x2);
  			
  			
  		 \draw[thick, ->] (-0.05, 0) -- (1, 0) node[below]{$\rndp$};
  	\draw[thick, ->] (0, -0.05) -- (0, 1) node[above] {$\scp$};
  	
  
    \fill[ domain=0.3:0.6, smooth, variable=\x, green!10!white,thick] plot ({\x}, {-5.555*(\x-0.3)*(\x-0.3) + 0.8}) -- (0.3,0.3);
    
  		\draw[name path=f, domain=0.3:0.6, smooth, variable=\x, green!50!black,thick] plot ({\x}, {-5.555*(\x-0.3)*(\x-0.3) + 0.8});
    
  	
  	\draw[dashed] (0,0) -| (x1);
  	\draw[dashed] (0,0) -| (x2);
  	 \draw[dashed] (0,0) |- (x1);
  	\draw[dashed] (0,0) |- (x2);
  	
  		\draw[-,thick,color=green!50!black] (0,0.8) -- (x2);
  	\draw[-,thick,color=green!50!black] (0.6,0.0) -- (x1);
  	 
  	\draw (x1) node[circle,fill=green, inner sep=1.5pt,draw=green!50!black] {};
  \draw (x2) node[circle,fill=green, inner sep=1.5pt,draw=green!50!black] {};
  \draw (x3) node[circle,fill=green, inner sep=1.5pt,draw=green!50!black] {};
  
  
  	
  	
\end{tikzpicture}
\caption{Solutions $\solutions$}
\label{fig:geom:solution}
\end{subfigure}
\begin{subfigure}{0.19\textwidth}
\centering
\begin{tikzpicture}[scale=2.5]	
     	
  	\node at (0.6,0.3) (x1) {};
  		\node at (0.3,0.8) (x2) {};
  	
  		\node at (0.5,0.57778) (x3) {};
  		
  					\draw[draw=none,pattern= bricks, pattern color=black!50] (1,0) rectangle (x1);
  					
  					\draw[draw=none,pattern= bricks, pattern color=black!50] (0,1) rectangle (x2);
  	
  		\draw[fill=green!10!white,draw=green] (0,0) rectangle (x1);
  			\draw[fill=green!10!white,draw=green] (0,0) rectangle (x2);
  			
  				\draw[fill=red!10!white,draw=red!0!white] (1,1) rectangle (x1);
  			\draw[fill=red!10!white,draw=red!0!white] (1,1) rectangle (x2);
  					\draw[fill=red!10!white,draw=red!0!white] (1,1) rectangle (x3);
  						\draw[draw=red,dashed] (1,1) rectangle (x1);
  			\draw[draw=red,dashed] (1,1) rectangle (x2);
  					\draw[draw= red,dashed] (1,1) rectangle (x3);
  					
  					\draw[draw=none,dashed,pattern=dots, pattern color=red] (1,1) rectangle (x1);
  					\draw[draw=none,dashed,pattern=dots, pattern color=red] (1,1) rectangle (x2);
  					\draw[draw=none,dashed,pattern=dots, pattern color=red] (1,1) rectangle (x3);
  					
  					
  			
  		 \draw[thick, ->] (-0.05, 0) -- (1, 0) node[below]{$\rndp$};
  	\draw[thick, ->] (0, -0.05) -- (0, 1) node[above] {$\scp$};
  	
  	
  	 \fill[draw=green!50!black] (x2.center) -- (x3.center) -- (x1.center);
    \fill[color=green!10!white] (x2.center) -- (x3.center) -- (x1.center) --  (0.3,0.3);
    
%  		\draw[name path=f, domain=0.3:0.6, smooth, variable=\x, green!50!black,thick] plot ({\x}, {-5.555*(\x-0.3)*(\x-0.3) + 0.8});
%    
  	
  	\draw[dashed] (0,0) -| (x1);
  	\draw[dashed] (0,0) -| (x2);
  	 \draw[dashed] (0,0) |- (x1);
  	\draw[dashed] (0,0) |- (x2);
  	
  		\draw[-,thick,color=green!50!black] (0,0.8) -- (x2);
  	\draw[-,thick,color=green!50!black] (0.6,0.0) -- (x1);
  	 
  	\draw (x1) node[circle,fill=green, inner sep=1.5pt,draw=green!50!black] {};
  \draw (x2) node[circle,fill=green, inner sep=1.5pt,draw=green!50!black] {};
  \draw (x3) node[circle,fill=green, inner sep=1.5pt,draw=green!50!black] {};
  \node[anchor=north] at (0.6,0) {$\rndopt$};
  		
  		\node[anchor=north] at (0.3,0) {$\rndmin$};
  		
  		\node[anchor=east] at (0,0.85) {$\scopt$};
  		
  		\node[anchor=east] at (0,0.3) {$\scmin$};
  
  	
  	
\end{tikzpicture}
\caption{Iterative construction}
\label{fig:geom:iterative}
\end{subfigure}
\begin{subfigure}{0.19\textwidth}
\centering
\begin{tikzpicture}[scale=2.5]	
     	
  	\node at (0.85,0.1) (x1) {};
  		\node at (0.1,0.85) (x2) {};
  		
  		\node at (0.55,0.55) (x3) {};
  		
  		\node[anchor=north] at (0.85,0) {$\rndopt$};
  		\node[anchor=north] at (0.55,0) {$\randomness_\delta$};
  		
  		\node[anchor=north] at (0.1,0) {$\rndmin$};
  		
  		\node[anchor=east] at (0,0.85) {$\scopt$};
  		\node[anchor=east] at (0,0.55) {$\scthreshold_\epsilon$};
  		
  		\node[anchor=east] at (0,0.1) {$\scmin$};
  	
  		\draw[fill=green!10!white,draw=green] (0,0) rectangle (x1);
  			\draw[fill=green!10!white,draw=green] (0,0) rectangle (x2);
  			
  			 \fill[draw=white,fill=green!10!white] (x1.center) -- (x2.center) -- (0.1, 0.1);
    
  			
  		 \draw[thick, ->] (-0.05, 0) -- (1, 0) node[below]{\phantom{$\rndp$}};
  	\draw[thick, ->] (0, -0.05) -- (0, 1) node[above] {\phantom{$\scp$}};
  	
  	\draw[dashed] (0,0) -| (x1);
  	\draw[dashed] (0,0) -| (x2);
  	 \draw[dashed] (0,0) |- (x1);
  	\draw[dashed] (0,0) |- (x2);
  	
  	 \draw[dashed] (0,0) |- (x3);
  	\draw[dashed] (0,0) -| (x3);
  	 
  	\draw (x1) node[circle,fill=green, inner sep=1.5pt,draw=green!50!black] {};
  	
  	\draw (x2) node[circle,fill=green, inner sep=1.5pt,draw=green!50!black] {};
  
  	
  	
  	\draw (x3) node[circle,draw,fill=white, inner sep=1.5pt,draw=black] {};
  	
\end{tikzpicture}
\caption{Regret-based ERCI}
\end{subfigure}
\begin{subfigure}{0.19\textwidth}
\centering
\begin{tikzpicture}[scale=2.5]	
     	
  	\node at (0.85,0.1) (x1) {};
  		\node at (0.1,0.85) (x2) {};
  		
  		
  		\node[anchor=north] at (0.85,0) {$\rndopt$};
  		
  		\node[anchor=north] at (0.1,0) {$\rndmin$};
  		
  		\node[anchor=east] at (0,0.85) {$\scopt$};
  		
  		\node[anchor=east] at (0,0.1) {$\scmin$};
  	
  		\draw[fill=green!10!white,draw=green] (0,0) rectangle (x1);
  			\draw[fill=green!10!white,draw=green] (0,0) rectangle (x2);
  			
  		 \draw[thick, ->] (-0.05, 0) -- (1, 0) node[below]{\phantom{$\rndp$}};
  	\draw[thick, ->] (0, -0.05) -- (0, 1) node[above] {\phantom{$\scp$}};
  	
  	\draw[dashed] (0,0) -| (x1);
  	\draw[dashed] (0,0) -| (x2);
  	 \draw[dashed] (0,0) |- (x1);
  	\draw[dashed] (0,0) |- (x2);
  	
  	 
  	\draw (x1) node[circle,fill=green, inner sep=1.5pt,draw=green!50!black] {};
  	
  	\draw (x2) node[circle,fill=green, inner sep=1.5pt,draw=green!50!black] {};
  	
  	\draw (x1) node[cross=2pt,color=blue] (c3) {};
  	
  	\draw (x2) node[cross=2pt,color=blue] (c3) {};
  
  \draw[black,->] (0.4,0.1) arc (0:20:0.3cm);
  
  		\node at (0.79,0.34) (x3) {};
  			\draw (x3) node[circle,fill=green, inner sep=1.5pt,draw=green!50!black] {};
  			
  	\draw (x3) node[cross=2pt,color=blue] (c3) {};
  	\draw [-] (0.1,0.1) -- (x3);
   \draw[black,->] (0.45,0.1) arc (0:40:0.35cm);
   
   \node at (0.63,0.54) (x4) {};
  			\draw (x4) node[circle,fill=green, inner sep=1.5pt,draw=green!50!black] {};
  			
  	\draw (x4) node[cross=2pt,color=blue] (c3) {};
  	\draw [-] (0.1,0.1) -- (x4);
   
   \draw[black,->] (0.5,0.1) arc (0:58:0.4cm);
    \node at (0.46,0.67) (x5) {};
  			\draw (x5) node[circle,fill=green, inner sep=1.5pt,draw=green!50!black] {};
  			
  	\draw (x5) node[cross=2pt,color=blue] (c3) {};
  	\draw [-] (0.1,0.1) -- (x5);
   \draw[black,->] (0.55,0.1) arc (0:75:0.45cm);
   \node at (0.278,0.776) (x6) {};
  			\draw (x6) node[circle,fill=green, inner sep=1.5pt,draw=green!50!black] {};
  			
  	\draw (x6) node[cross=2pt,color=blue] (c3) {};
  	\draw [-] (0.1,0.1) -- (x6);
  	
   \draw[black,->] (0.23,0.58) arc (75:65:0.5cm);
   \node at (0.386,0.715) (x7) {};
  			\draw (x7) node[circle,fill=green, inner sep=1.5pt,draw=green!50!black] {};
  			
  	\draw (x7) node[cross=2pt,color=blue] (c3) {};
  	\draw [-] (0.1,0.1) -- (x7);

   
   \draw[-] (x1) -- (x3) -- (x4) -- (x5) -- (x7) -- (x6) -- (x2);
   
   \node[anchor=west,xshift=0.3em] at (x3) {\tiny{1:$\lambda{=}1$}};
   \node[anchor=west,xshift=0.3em] at (x4) {\tiny{2:$\lambda{=}2$}};
   \node[anchor=west,xshift=0.3em] at (x5) {\tiny{3:$\lambda{=}4$}};
   \node[anchor=west,yshift=0.3em,xshift=0.2em] at (x7) {\tiny{5:$\lambda{=}6$}};
   \node[anchor=west,yshift=0.4em,xshift=0.1em] at (x6) {\tiny{4:$\lambda{=}8$}};
  	
\end{tikzpicture}
\caption{Rationality-based algorithm}
\end{subfigure}

\caption{Geometric interpretation of the ERCI problem for some fixed SG.}
\end{figure*}
It is convenient to consider this front geometrically.
To begin, given a fixed ERCI instance, a scheduler $\sched$
\emph{induces} a point $x_\sched$:
\begin{equation}
  x_\sched \eqdef \Big\langle \Pr(X_\varphi \mid \sched), H(\sched) \Big\rangle \in [0,1] \times [0,\infty).  
\end{equation}
To ease notation, for $x_\sched = \langle \scp,\rndp \rangle$ we use
$\scp_\sched \eqdef \scp$ and $\rndp_\sched \eqdef \rndp$. Next, we
partially order these points via the standard product ordering:
\begin{equation}
  \langle \scp,\rndp \rangle \preceq \langle \scp',\rndp' \rangle \quad\text{ iff }\quad \scp \leq \scp' \wedge \rndp \leq \rndp'.
\end{equation}

We say that $\pOneSched$ \emph{guarantees} a point $x_\pOne \eqdef
\langle \scp, \rndp \rangle$, if for every policy $\pTwoSched$, using
$\sched = \langle \pOneSched, \pTwoSched \rangle$, we have
$\scp_\sched \geq \scp$ and $\rndp_\sched \geq \rndp$. Thus, a point
is guaranteed if no matter what policy $\pTwo$ uses, $x_\sched$ will
induce a point no worse w.r.t.\ to either randomization or performance
than $x_\pOne$. 
We define
\emph{the set of guaranteed points} for a scheduler $\pOneSched$:
\begin{equation}\label{eq:guaranteed}
  \solutions[\pOneSched] \eqdef \{ \langle \scp, \rndp \rangle \mid  \pOneSched \text{ guarantees } \langle \scp, \rndp \rangle \}.
\end{equation}
We observe that guaranteed points are
downward closed, i.e., if $\pOneSched$ guarantees $x$ and $x' \preceq x$,
then $\pOneSched$ guarantees $x'$.
\begin{example}
Consider Fig.~\ref{fig:geom:guarantee}. We fix $\pOneSched$ and draw all points induced by $\sched = \langle \pOneSched, \pTwoSched \rangle$ when varying $\pTwoSched$ in the blue hatched area. We take the minimal randomness $\rndp$ and the minimal performance $\scp$. The points in the downward closure  of $\langle \scp, \rndp \rangle$ (green circle) are the guaranteed points for $\pOneSched$ in the green solid area.	
We notice the gap between both areas: While the performance and randomization may be better than the optimum that $\pOne$ can guarantee, it cannot guarantee a higher randomization \emph{and} performance simultaneously, as  the $\pTwo$-player would have a counter-policy violating either the performance \emph{or} the randomization.
\end{example}

Points guaranteed by some $\pOneSched$ are called
\emph{achievable}. Thus, the achievable points are: $ \solutions =
\bigcup_{\pOneSched} \solutions[\pOneSched]$.  Importantly, the ERCI problem is satisfiable iff $\langle \scthreshold,
\randomness \rangle$ is achievable. 
Thus, to solve ERCI instances, we start by characterizing
$\solutions$. We start by observing the $\solutions$ is convex (proof in Sec~\ref{sec:proofs}).


\begin{proposition}\label{prop:convex}
  The set of achievable points, $\solutions$, is convex. 
\end{proposition}

\footnotetext{
  That is, $y, y' \in Y$ implies for
  every $w \in [0,1]$ that $w \cdot y + (1-w) \cdot y \in Y$
}

Next, because $\solutions$ is downward closed, it
suffices to study the ``maximal'' or non-dominated points.  Precisely,
we say that a point $x$ is \emph{dominated} by $x'$ if $x \prec
x'$, i.e., if $x \preceq x' \wedge x \neq x'$.
The Pareto-front $\pareto{\solutions}$ of $\solutions$ is then the set of non-dominated guaranteed points,
\begin{equation}
  \pareto{\solutions} \eqdef \{ x \in \solutions \mid \forall x' \in \solutions, x \not\prec x'  \}.  
\end{equation}
\noindent
\begin{mdframed}
Importantly, it holds that the ERCI problem is satisfiable iff there exists a  $x \in \pareto{\solutions}$ such that $\langle \scthreshold, \randomness \rangle \preceq x$.    
\end{mdframed}
\begin{example}
	The set $\solutions$ illustrated in Fig.~\ref{fig:geom:solution} is obtained by taking the union of guaranteed points, and can be characterized by the set of points on the Pareto-front: This is the curved border between the green and white area, in particular the three green dots are on the Pareto-front. Any ERCI instance with $\langle \scthreshold, \randomness \rangle$ in the green area is satisfiable.
\end{example}


\noindent
Approximating the Pareto-front gives a natural approximation
scheme for ERCI instances: For any subset $\pareto{} \subseteq
\pareto{\solutions}$,
\begin{enumerate}
\item If there exists an $x \in \pareto{}$ such that
$\langle \scthreshold, \randomness \rangle \preceq x$, then the ERCI
Problem must be satisfiable, $x$ is a witness to satisfiability.
\item If there exists an $x \in \pareto{}$
such that $x \prec \langle \scthreshold, \randomness \rangle$ then the
ERCI problem is not satisfiable, and $x$ is a witness to unsatisfiability.
\end{enumerate}
Due to convexity, we may speed up the search for satisfiability: If there exist $x_1, x_2 \in \pareto{}$ such that $\langle \scthreshold, \randomness \rangle \prec \big(w \cdot x_1 + (1-w) \cdot x_2\big)$, we call $x_1,x_2$ a witness-pair.

\begin{example}
\label{ex:approximation}
	Consider Fig.~\ref{fig:geom:iterative}. We have found three points on the Pareto-curve, and already have a good impression of the trade off between randomization and performance. In particular, the green area is definitively a subset of $\solutions$: It exploits the downward closure and the convexity of $\solutions$. The red (dotted) part contain the points on the Pareto curve in their downward closure, thus they cannot be part of the Pareto curve themselves.
	Furthermore, the topmost point on the Pareto curve was obtained by maximizing performance (and optimizing randomization only as a secondary objective). Thus, by construction, the bricked area at the top is not satisfiable. Analogously, the bricked area at the right reflects non-achievable entropy. 
\end{example}

Thus a key algorithmic question in ERCI is how to efficiently explore
and approximate the Pareto front $\pareto{\solutions}$. As first steps, we find the two special points induced by (1) optimizing performance and only then randomization (the topmost green point in the figures) and (2) optimizing randomization and only then performance (the rightmost green point). 
As we have seen, these restrict the domain in which we can actually trade performance for randomness. 
We define 
$\rndopt \eqdef \max \{ \rndp \mid \exists \scp \text{ s.t. } \langle \scp, \rndp \rangle \in \solutions  \} $, i.e., the largest randomness that can be guaranteed by any $\pOne$-policy. 
Likewise, we define 
$\scopt \eqdef \max \{ \scp \mid \exists \rndp \text{ s.t. } \langle \scp, \rndp \rangle \in \solutions  \} $, i.e., the largest performance that can be guaranteed by any $\pOne$-policy. 
Then, we define 
$\scmin \eqdef \max \{ \scp \mid \langle \scp, \rndopt \rangle  \in \solutions \}$, the best performance that $\pOne$ can guarantee while guaranteeing optimal randomness. 
Likewise, we define  the analogous $\rndmin \eqdef \max \{ \scp \mid \langle \scp, \rndopt \rangle  \in \solutions \}$.
We thus obtain two points on the Pareto-curve: $\langle \scmin, \rndopt \rangle$ and $\langle \scopt, \rndmin \rangle$, and intuitively, we can trade between these two points following the Pareto curve.

%\begin{remark}[Regret Based ERCI]

Often, rather that fixing $\scthreshold$ and $\randomness$ a
  priori, one seeks achieve some percentage of the achievable soft
  constraint or causal entropy measure.  We
  re-parameterize ERCI as follows:
  \begin{equation}
    \scthreshold_\epsilon \eqdef \epsilon \cdot (\scopt - \scmin) + \scmin
    \hspace{1em}
    \randomness_\delta \eqdef  \delta \cdot (\rndopt - \rndmin) + \rndmin
  \end{equation}
  where $\epsilon, \delta \in [0, 1]$. We call this version of ERCI \emph{regret-based}. Geometrically, after computing $\scopt$ and $\rndopt$, we know that the left triangle in Fig.~\ref{fig:geom:regret} is definitively satisfiable, and the regret-based ERCI asks whether the white circle is also satisfiable (where the point of the white point is given by $\epsilon$ and $\delta$. %As we shall later see, these maximum quantities are
 % directly computed in our proposed algorithm. 
%\end{remark}

Finally, it is helpful to think about the Pareto-curve as a function of randomization in this reparameterization.  We define a characteristic function which given a target
performance ratio, $\epsilon$, yields the optimal randomness ratio,
$\delta$:
\begin{equation}
  \begin{split}
    & \solfuncp\colon [0,1] \rightarrow [0, 1]    \\
    & \solfuncp(\delta) = \max_\epsilon \{ \randomness_\delta \mid \langle
    \scthreshold_\epsilon, \randomness_\delta \rangle \in \solutions \} 
  \end{split}
\end{equation}
\begin{proposition}\label{prop:monotone}
  $\solfuncp$ is continuous and (strictly) decreasing.
\end{proposition}
 We shall temporarily post-pone the proof of
\propref{monotone}. For now, one case observe that
(non-strict) monotone decreasing follows directly from convexity and
using the adequate domains.
Finally, the set  $\solutions$ is (in general) \emph{not} a finite polytope -- the MDP in Fig.~\ref{fig:minimal:mdp} serves as an example. Nevertheless,  $\solutions$ can be well approximated with finitely many vertices, see Ex.~\ref{ex:approximation}.

With these facts, we are now well-equiped to develop the algorithms in Sec.~\ref{sec:mdps} for MDPs and Sec.~\ref{sec:sgs} for SGs.

%%% Local Variables:
%%% mode: latex
%%% TeX-master: "main"
%%% End:
