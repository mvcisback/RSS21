
\section{The Control Improvisation Problem for MDPs}
\label{sec:mdps}

We present an algorithm for the control improvisation problem for
MDPs. The key insight is to rephrase the tradeoff between
randomisation and performance as a degree in rationality $\rat$ of the
policy. Intuitively, a rationality of $\rat = \infty$ means that we
focus completely on the performance criterion, and rationality $\rat =
0$. We can then reuse ideas from maximum entropy inference from
specifications~\cite{DBLP:conf/cav/Vazquez-Chanlatte20}.  In the next
section, we extend this idea to SGs.

\subsection{Rationality}

\noindent
In context of MDPs, the maximum causal entropy policy for a particular
reward (here performance) is given by a smooth variant of the Bellman
equations~\cite{mceThesis}. Namely, let $\smoothmax{}$ denote the
log-sum-exp operator, i.e., $\smoothmax{X} \eqdef \log \left(
\sum_{x\in X} e^x \right)$. For each rationality $\lambda \in [0,
\infty)$, we define a policy,
 \begin{align}
   &\sched_\rat(s\mid \act) = \log( Q_\rat(s,\act) - V_\rat(s))  \\
	& V_\rat(s) \colonequals  \begin{cases} \smoothmax_{\act \in \EnAct(s)}{\{  Q_\rat(s,\act) \}} & \text{if }s \not\in \{ \target, \sink \}, \\ \lambda  \cdot \indicator{s = \target} & \text{otherwise.}  \end{cases}\\ 
	& Q_\rat(s, \act) \colonequals \sum_{s'} P(s,\act,s') \cdot V_\rat(s').
 \end{align}
 for each $s \in S \setminus \{ \sink, \target \}$ and $\act \in \EnAct(s)$ and
\sj{Need to put in enact}
To ease notation, we denote $x_\rat \colonequals x_{\sched_\rat}, \scp_\rat \colonequals \scp_{\sched_\rat}, \rndp_\rat \colonequals \rndp_{\sched_\rat}$.

\begin{proposition}
$\scp_\rat$ is monotonically increasing in $\rat$ and $\rndp_\rat$ is monotonically decreasing in $\rat$.	
\end{proposition}
As a consequence, we can use $\rat$ to explore the Pareto-front. 


{\color{red}Need to talk about completness}
\color{black!50}
We consider $P$ as being defined using an auxiliary notion of environment actions $\Act_E$, a deterministic environment transition relation $P_{\hat{i}}\colon S_i \times \Act_i \rightarrow A_E$ and a (memoryless, randomized) environment-scheduler $A_E \rightarrow \Distr(S)$.



It is convenient to talk about a third player, $\mathsf{rnd}$, that owns all states with $|\Act|$\sj{needs available actions}



\color{black}

\subsection{Pareto-exploration}



%%% Local Variables:
%%% mode: latex
%%% TeX-master: "main"
%%% End:
