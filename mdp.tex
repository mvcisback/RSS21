\section{The Control Improvisation Problem for MDPs}
\label{sec:mdps}

We present an algorithm for the control improvisation problem for
MDPs. Recall that an MDP is a stochastic game with a 
The key idea is to rephrase the tradeoff between
randomisation and performance as a degree in rationality $\rat$ of the
policy. Intuitively, a rationality of $\rat = \infty$ means that we
focus completely on the performance criterion, and rationality $\rat =
0$. We can then reuse ideas from maximum entropy inference from
specifications~\cite{DBLP:conf/cav/Vazquez-Chanlatte20}.  In the next
section, we extend this idea to SGs.

\subsection{Rationality}

\noindent
In context of MDPs, the maximum causal entropy policy consistent with
an expected reward (here performance, $\scthreshold$) is given by a
smooth variant of the Bellman equations~\cite{mceThesis}. Namely, let
$\smoothmax{}$ denote the log-sum-exp operator, i.e., $\smoothmax(X)
\eqdef \log \left( \sum_{x\in X} e^x \right)$. For each rationality
$\rat \in [0, \infty)$, we define a policy,
 \begin{align}
   &\sched_\rat(s\mid \act) \eqdef \exp( Q_\rat(s,\act) - V_\rat(s))  \\
   & V_\rat(s) \eqdef  \begin{cases}
     \lambda  \cdot \indicator{s = \target} & \text{if }s \in \{ \target, \sink \},\\
     \smoothmax_{\act \in \EnAct(s)}{  Q_\rat(s,\act) } & \text{otherwise.}
   \end{cases}\\ 
	& Q_\rat(s, \act) \eqdef \sum_{s'} P(s,\act,s') \cdot V_\rat(s').
 \end{align}
\sj{Need to put in enact}
To ease notation, we denote $x_\rat \colonequals x_{\sched_\rat},
\scp_\rat \colonequals \scp_{\sched_\rat}, \rndp_\rat \colonequals
\rndp_{\sched_\rat}$. As previously alluded, the key property
is that $\sched_\rat$ is the \emph{unique} maximum causal entropy policy
such that $\Pr(\varphi) = \scp_\rat$~\cite{DBLP:conf/cav/Vazquez-Chanlatte20}.

In terms of the machinery developed in the previous section, this
family serves to index the Pareto-Front, $\pareto{\solutions}$.  As a
consequence, we can use $\rat$ to explore the Pareto-front.  To see
this, first observe the following easily verified proposition.

\begin{proposition}
  $\scp_\rat$ is smoothly and (strictly) monotonically increasing in $\rat$ and $\rndp_\rat$
  is smoothly (strictly) monotonically decreasing in $\rat$.
\end{proposition}

Intuitively, as $\rat$ approaches $0$, $\sched_\rat$ approaches the
uniform distribution over \emph{all available actions}. Note that this
policy maximizes (causal) entropy, and thus $\rndopt = \rndp_0$.
Similarly, as $\rat$ approaches $\infty$, $\sched_\rat$ selects (uniformly) from
actions \emph{that maximize performance}. Thus, $\scopt = \scp_\infty$.

\subsection{Pareto-exploration}
As a consequence, we can use $\rat$ to explore the Pareto-front.
In particular, assuming $\scopt, \rndopt \neq 0$ (which would
otherwise yield trivial $\solutions$ and $\pareto{\solutions}$), one
can define $\epsilon_\rat \eqdef \nicefrac{\scp_\rat}{\scp_\infty}, \delta_\rat \eqdef \nicefrac{\rndp_\rat}{\rndp_0}$. Then, because
$\sched_\rat$ maximizes randomness given a target performance, one derives:
\begin{equation}
  \solfuncp\left(\epsilon_\rat\right) = \delta_\rat.
\end{equation}
The key algorithmic idea is thus to strategically evaluate a sequence
of rationality coefficients to yield (input, output) pairs for
$\solfuncp$. Due to convexity, the convex hull this sequence of
rationality-indexed points (and the origin) gradually refines a
polygonal approximation of $\solutions$, and thus the Pareto
Front. This approximation, $\hat{\solutions}$, is refined until
either:
\begin{enumerate}
\item $\langle \scthreshold, \randomness \rangle \in \hat{S}$ proving
  $\langle \scthreshold, \randomness \rangle \in \solutions$.
\item A $\rat$ is found such that
  $x_{\rat} \prec \langle \scthreshold, \randomness \rangle$, proving
  $\langle \scthreshold, \randomness \rangle \notin \solutions$.
\end{enumerate}
Next, to extract an improviser, observe that because
$\hat{\solutions}$ is a convex polygon, if $\langle \scthreshold,
\randomness \rangle \in \hat{S}$, then there must two corners of
$\hat{\solutions}$ indexed by $\rat_1$ and $\rat_2$, that form a
triangle with $(0, 0)$ containing $\langle \scthreshold, \randomness
\rangle$. Thus, as in the convexity proof, there must be a convex
combination of $q\cdot x_{\rat_1} + \bar{q}\cdot x_{\rat_2}$ that
dominates $\langle \scthreshold, \randomness \rangle$. Therefore, the
following policy solves the ERCI instance:
\begin{equation}\label{eq:4}
  \sigma^*_\pOne(a\mid s) \eqdef q\cdot \sched_{\rat_1}(a\mid s) + \bar{q} \cdot \sched_{\rat_2}(a \mid s)
\end{equation}

\mypara{Approximation Sequence} The final algorithmic question for
MDPs is then: what order should one evaluate rationality coefficients.
We propose a three staged sequence: (i) Compute $x_\rat$ for the end
points $\rat \in \{0, \infty\}$.  (ii) Double $\rat$ until $h_\rat \leq
\randomness$, yielding $\rat_1\ldots \rat_j$, where $\rat_1 = 1$.
(iii) Binary search for $\rat \in [\rat_{j-1}, \rat{j}]$.
\begin{mdframed}
  Our approximation scheme yields a semi-decision process which halts
  iff either (a) $\langle \scthreshold, \randomness \rangle$ is
  bounded away from $\pareto{\solutions}$ \emph{or} (b)
  $\langle \scthreshold, \randomness \rangle$ is visited by
  $x_{\rat_i}$.
\end{mdframed}
Next, observe that given a maximum resolution, $\kappa$, in
rationality, this approximation scheme becomes linear in the MDP size
and logarithmic in the final rationality coefficient $\rat_*$ and the
resolution $\kappa$, i.e., the run-time is,
\begin{equation}
  O\bigg(\hspace{-1.4em}\underbrace{|\sg|}_{\text{Bellman Backup}}\hspace{-1.4em}\cdot\overbrace{\log(\rat_*)}^{\text{Doubling Phase}}\cdot\underbrace{\log(\nicefrac{1}{\kappa})}_{\text{Binary Search}}\bigg)
\end{equation}
Finally, before generalizing to stochastic games, we observe that in
practice, $\sched_{100} \approx \sched_\infty$, and one can often take
$\rat_* \leq 100$.

%%% Local Variables:
%%% mode: latex
%%% TeX-master: "main"
%%% End:
