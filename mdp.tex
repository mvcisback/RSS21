\section{The Control Improvisation Problem for MDPs}
\label{sec:mdps}

We present an algorithm for the control improvisation problem for
MDPs, which in the next section, will serve as a subroutine for an algorithm
on SGs. To start, recall that an MDP is a stochastic game with no action choices for the environment, i.e., the environment is purely stochastic. 
We want to instantiate the approximation scheme from the previous section. In particular, that means that we need to find points on the Pareto curve $\pareto{\solutions}$ and thus incrementally build up $\pareto{} \subseteq \pareto{\solutions}$. 
 
\subsection{Rationality}
The key idea for finding points on the Pareto-curve is to rephrase the
trade-off between randomization and performance as a degree in
rationality $\rat$ of the policy.  Formally, the rationality
corresponds to the following scalarizion of our multi-objective
problem~\cite{DBLP:journals/corr/abs-1805-00909},
\begin{equation}
  \label{eq:scalarization}
  J_\rat(\sched) \eqdef \Big\langle 1, \rat\Big\rangle \cdot \Big\langle\rndp_\sched, \scp_\sched\Big\rangle.
\end{equation}
In context of MDPs, the \textbf{unique} policy that
optimizes~\eqref{eq:scalarization} is given by a smooth variant of the
Bellman equations~\cite{mceThesis, DBLP:conf/cav/Vazquez-Chanlatte20}. Namely, let $\smoothmax{}$ denote
the log-sum-exp operator, i.e.,
$\smoothmax(X) \eqdef \log \left( \sum_{x\in X} e^x \right)$. For each
rationality $\rat \in [0, \infty)$, we define a policy $\sched_\rat$
-- using $s = \last{\path}$ -- as follows:
 \begin{align}
   &\sched_\rat(\act \mid s) \eqdef \exp( Q_\rat(s,\act) - V_\rat(s))  \\
   & V_\rat(s) \eqdef  \begin{cases}
     \lambda  \cdot \indicator{s = \target} & \text{if }s \in \{ \target, \sink \},\\
     \smoothmax_{\act \in \EnAct(s)}{  Q_\rat(s,\act) } & \text{otherwise.}
   \end{cases}\label{eq:mdp:v}\\ 
	& Q_\rat(s, \act) \eqdef \sum_{s'} P(s,\act,s') \cdot V_\rat(s').
 \end{align}

To ease notation, we denote $x_\rat \eqdef x_{\sched_\rat},
\scp_\rat \eqdef \scp_{\sched_\rat}, \rndp_\rat \eqdef
\rndp_{\sched_\rat}$. 
 % As previously alluded at the start of the subsection, the key property
% is that $\sched_\rat$ is the \emph{unique} maximum causal entropy policy
% such that $\Pr(\varphi) = \scp_\rat$~\cite{mceThesis}.

Intuitively, as $\rat$ approaches $0$, $\sched_\rat$ approaches the
uniform distribution over \emph{all available actions}. Note that this
policy maximizes (causal) entropy, and thus $\rndopt = \rndp_0$.  For
$\lambda \rightarrow \infty$, this variant of the Bellman equations
coincides with the standard Bellman
equations~\cite{DBLP:books/wi/Puterman94}, where $\sched_\rat$ selects
(uniformly) from actions \emph{that maximize performance}.
Furthermore, the monotonicity and smoothness of the above Bellman
equations yields the following proposition.
\begin{proposition}
  $\scp_\rat$ is  continuously (and strictly) increasing in $\rat$ and $\rndp_\rat$
  is smoothly (and strictly) decreasing in $\rat$.
\end{proposition}
In terms of $\solfuncp$, we can define:
\begin{equation}
  \epsilon_\rat \eqdef \frac{\scp_\rat - \scp_0}{\scp_\infty} + \scp_0
  \hspace{1em}
\delta_\rat \eqdef \frac{\rndp_\rat - \rndp_\infty}{\rndp_0} +
\rndp_{\infty}. 
\end{equation}

Then, because $\sched_\rat$ maximizes randomness
given a target performance, one derives:
\begin{equation}
  \solfuncp\left(\delta_\rat\right) = \epsilon_\rat.
\end{equation}

The key idea now is to instantiate the approximation scheme for the Pareto front by varying $\rat$.\footnote{Assuming $\scopt, \rndopt \neq 0$ (which would
otherwise yield trivial $\solutions$ and $\pareto{\solutions}$)}.
In particular, we construct $\pareto{} = \{ x_\rat \mid \rat \in \{ \rat_1, \rat_2, \hdots \} \}$ until $\pareto{}$ contains a witness to either realizability or unrealizability of the ERCI instance. 
In the remainder of this section, we improve upon randomly selecting values for $\rat$.



%
%and by varying $\rat$ we can explore the Pareto front. First observe the following easily verified proposition.
%\begin{proposition}
%  $\scp_\rat$ is smoothly and (strictly) monotonically increasing in $\rat$ and $\rndp_\rat$
%  is smoothly (strictly) monotonically decreasing in $\rat$.
%\end{proposition}

\subsection{Targeted Pareto-exploration}
The key ingredient to improve upon arbitrarily selecting $\rat_1, \hdots \rat_i$ is to exploit additional structure of the rationality.  
%The key algorithmic idea is thus to strategically evaluate a sequence
%of rationality coefficients to yield (input, output) pairs for
%$\solfuncp$. Due to convexity, the convex hull this sequence of
%rationality-indexed points (and the origin) gradually refines a
%polygonal approximation of $\solutions$, and thus the Pareto
%Front. This approximation, $\hat{\solutions}$, is refined until
%either:
%\begin{enumerate}
%\item $\langle \scthreshold, \randomness \rangle \in \hat{\solutions}$ proving
%  $\langle \scthreshold, \randomness \rangle \in \solutions$.
%\item A $\rat$ is found such that
%  $x_{\rat} \prec \langle \scthreshold, \randomness \rangle$, proving
%  $\langle \scthreshold, \randomness \rangle \notin \solutions$.
%\end{enumerate}
%
%
%Next, to extract an improviser, observe that because
%$\hat{\solutions}$ is a convex polygon, if $\langle \scthreshold,
%\randomness \rangle \in \hat{S}$, then there must two corners of
%$\hat{\solutions}$ indexed by $\rat_1$ and $\rat_2$, that form a
%triangle with $(0, 0)$ containing $\langle \scthreshold, \randomness
%\rangle$. Thus, as in the convexity proof, there must be a convex
%combination of $q\cdot x_{\rat_1} + \bar{q}\cdot x_{\rat_2}$ that
%dominates $\langle \scthreshold, \randomness \rangle$. Therefore, the
%following policy solves the ERCI instance:

%\mypara{Approximation Sequence} 
%The final algorithmic question for
%MDPs is then: what order should one evaluate rationality coefficients.
We propose a three staged sequence: (i) Compute $x_\rat$ for the end
points $\rat \in \{0, \infty\}$.  (ii) Double $\rat$ (starting at $\lambda=1$) until $h_\rat \leq
\randomness$, yielding $\rat_1\ldots \rat_j$.
(iii) Binary search for $\rat \in [\rat_{j-1}, \rat_{j}]$. We illustrate the idea in Fig.~\ref{fig:geom:doubling}.

The algorithm terminates almost surely, that is: 
the algorithm halts if $\langle \scthreshold, \randomness \rangle$ is
not on $\pareto{\solutions}$ (or if we happen to exactly hit $\langle \scthreshold, \randomness\rangle$ by selecting some rationality $\rat$).
As the Pareto Front has
measure 0, we argue that not halting is thus merely a technical concern, as a
small perturbation to the ERCI instance (i.e. a \emph{smoothed
analysis}~\cite{SmoothedAnalysis}) on $\sg$ admits decidability. 
\begin{mdframed}
  Our approximation scheme yields a semi-decision process which halts
  iff either (a) $\langle \scthreshold, \randomness \rangle$ is
  bounded away from $\pareto{\solutions}$ \emph{or} (b)
  $\langle \scthreshold, \randomness \rangle$ is dominated by
  $x_{\rat_i}$.
\end{mdframed}
Next, observe that if we terminate the binary search when the search
region is smaller than $\Delta$, this approximation scheme becomes linear in the MDP size
and logarithmic in the final rationality, $\rat_*$, and the
resolution, $\Delta$, i.e., the run-time is,
\begin{equation}
  O\bigg(\hspace{-1.4em}\underbrace{|\sg|}_{\text{Evaluate Point $x_\rat$}}\hspace{-1.4em}\cdot\overbrace{\log(\rat_*)}^{\text{Doubling Phase}}\cdot\underbrace{\log(\nicefrac{1}{\Delta})}_{\text{Binary Search}}\bigg)
\end{equation}
Finally, before generalizing to stochastic games, we observe that in
practice, $\rat = 100$ yields a nearly optimal policy, and thus one
can often assume $\rat_* \leq 100$ in our run-time analysis.

%%% Local Variables:
%%% mode: latex
%%% TeX-master: "main"
%%% End:
