\section{Discussion, Related work and Conclusion}

\subsection{Control Improvisation in the Literature}

In this section, we briefly compare ERCI with other forms of control
improvisation. Firstly, we observe that general Control Improvisation
has been proposed in stochastic environments for lane
changing~\cite{DBLP:conf/cdc/GeM18} and imitating power usage in
households~\cite{DBLP:conf/iotdi/AkkayaFVDLS16}. However, in those
both settings, the randomness constraint is phrased as an upper-bound
on the probability of indefinitely-long paths. Consequently, those
randomness constraints are trivially satisfied and the obtained
policies are deterministic.  In comparison, we consider the synthesis
of policies that necessarily randomize in presence of stochastic
behavior in the environment. The closest prior work is to ours
is Reactive Control Improvisation (RCI) for (deterministic) 2-player games
~\cite{DBLP:conf/cav/FremontS18}. As in ERCI, this work features
a three kinds of constraints, hard, soft, and randomness. As in ERCI,
RCI can be preprocessed resulting in the follow core problem.
\begin{mdframed}
  \textbf{The Core RCI Problem}: Given an finite acyclic
  (deterministic) SG $\sg$, with sink states, $\target$ and $\sink$,
  and thresholds $\scthreshold \in (0,1)$ and
  $\randomness \in [0,\infty)$, find a $\pOne$-policy $\pOneSched$
  such that for every $\pTwo$-policy $\pTwoSched$
  \begin{enumerate}
  \item (\emph{soft constraint)}
    $\Pr(\last{\xi} = \target \mid \sched) \geq \scthreshold$,
  \item (\emph{randomness})
    $\max_\xi{\Pr(\path \mid \sched)} \leq \ppthreshold$,
   \end{enumerate}
   where $\sched = \langle \pOneSched, \pTwoSched \rangle$.
 \end{mdframed}
While RCI is only applied to deterministic SGs in
\cite{DBLP:conf/cav/FremontS18}, there is nothing in the definition
that prevents its application to the general class of
SGs\footnote{However, this does not mean that the algorithm to compute
a solution carries over to the general case}.  We observe that then,
the only difference between ERCI and RCI is that we use causal entropy
rather than an upper bound on the probability of a path to enforce
randomness.  The follow example motivates the choice of causal entropy
to formalize randomness in SGs.
\begin{figure}
\begin{tikzpicture}
        \node[sstate, initial, initial text=] (s0) {$s_0$};
  	\node[actnode,right=of s0, xshift=-2em] (e0) {};
        
	\node[sstate,right=of s0, yshift=2.5em] (s2) {$s_2$};
        \node[sstate,above=0.4cm of s2] (s1) {$s_1$};
	\node[below=0.1cm of s2] (sdots) {$\vdots$};
	\node[sstate,below=0.1cm of sdots] (sn) {$s_n$};
        
	\node[sstate,right=of s1] (t1) {$t_1$};
	\node[right=of t1] (tdots) {$\hdots$};
	\node[sstate,right=of tdots] (tn) {$t_m$};
	\node[sstate, right=2.2cm of e0] (u) {$u$};
	\node[sstate, right=of u, yshift=1em] (v1) {$v_1$};
	\node[sstate, right=of u, yshift=-1em] (v2) {$v_2$};
	\node[right=0.5cm of v1] {$\hdots$};
	\node[right=0.5cm of v2] {$\hdots$};
	
	
	
	\draw[->] (t1) -- node[elab] {$1$} (tdots);
	\draw[->] (tdots) -- node[elab] {$1$} (tn);

	\draw[->] (s0) -- node[elab] {} (e0);
        
	\draw[->] (e0) -- node[elab] {$\nicefrac{1}{n}$} (s1);
	\draw[->] (e0) -- node[elab,below] {$\nicefrac{1}{n}$} (s2);
	\draw[->] (e0) -- node[elab,below] {$\nicefrac{1}{n}$} (sn);
	
	\draw[->] (s1) -- node[elab,above] {$1$} (t1);
	\draw[->] (s2) -- node[elab,near end,above] {$1$} (u);
	\draw[->] (sn) -- node[elab,near end,below] {$1$} (u);
	
	\draw[->] (u) -- node[actnode] {} node[elab,above, near start] {$a$}  node[elab,above, near end] {$1$} (v1);
	\draw[->] (u) -- node[actnode] {} node[elab,below, near start] {$b$} node[elab,below, near end] {$1$} (v2);
	
	
\end{tikzpicture}
\caption{Example Illustrating the problem with RCI in stochastic environments.\label{fig:drciOnSgs}}
\end{figure}

\begin{example}
  Consider the SG (actually, an MDP) in Fig.~\ref{fig:drciOnSgs}.
  First consider that under each scheduler, the path from $s_0$ to
  $t_m$ has probability $\nicefrac{1}{n}$. In particular, this means
  that a feasible RCI instance (applied to an SG) must have
  $\ppthreshold \geq \nicefrac{1}{n}$. At the same time, every path in
  the SG already has probability at most $\nicefrac{1}{n}$, and thus,
  every schedulder that satisfies the randomness constraint for
  $\delta = 1$ satisfies it for any
  $\ppthreshold \geq \nicefrac{1}{n}$. Thus, for this MDP, RCI fails
  enforce any randomization in the $\pOne$-policy. By contrast, a
  causal entropy constraint will continuously trade off randomness
  for performance, even in this setting.
\end{example}

% On the other hand, for deterministic SGs, all randomization is due to
% the random behavior of $\pOne$. 

% Roughly speaking, every
% (controllable) path is then just good (i.e., ending in a target state)
% or bad (i.e., ending in a sink state). The randomization criterion
% then may just ensure that we do not select the same path with too much
% probability, which intuitively means that the $\pOne$-player must
% ensure in every step that there are a sufficient number of paths from
% the next state (no matter what $\pTwo$-player does).  As long as there
% exist a sufficient number of paths, randomizing (uniformly) over those
% paths leads to a solution.

On the other hand, for deterministic SGs, the set of feasbile ERCI and
RCI instances coincide!
\begin{theorem}
  For any deterministic SG with some path sets $X_\varphi$, $X_\psi$
  and threshold $\scthreshold$, there exists an $\pOne$-policy solving
  the DRCI problem with threshold $\ppthreshold$ iff there exists a
  $\pOne$-policy solving the ERCI problem with threshold
  $\randomness = f(\ppthreshold)$ for an adequate (computable) $f$.
\end{theorem}
\begin{proof}[Proof Sketch]
  First, observe that as in ERCI, for RCI, one can w.o.l.g. assume
  $\pTwoSched$ is deterministic~\cite{DBLP:conf/cav/FremontS18}.
  Thus, given some worst-case adversary (for randomness and
  performance) all randomization is due to the behavior of
  $\pOne$. Assuming a non-trivial solution set (and thus
  $\scopt \neq 0$), this implies that $\pOne$ can \emph{first} decide
  whether to meet the soft constraint and \emph{then} choose actions
  weighted by the number of guaranteed ways to reach $\target$ in the
  corresponding subtree. Now observe when planning against the
  min-causal entropy adversary, $\sched_*^\pTwo$, the deterministic SG
  reduces to a deterministic MDP. In deterministic MDPs, the action
  sequence uniquely determines the state sequence, and thus the causal
  entropy reduces to the entropy of the actions and states, i.e., the
  paths, $\xi$.  Thus, treating $\xi$ as a random variable and
  case splitting on winning yields:
  \begin{equation}\label{eq:cond_win}
    \begin{split}
      h_{\langle \sched_\pOne, \sched^*_\pTwo\rangle} &= \scthreshold\cdot H(\xi \mid \text{win}) + (1 - \scthreshold)\cdot H(\xi \mid \neg \text{win})\\
      &\leq \scthreshold\cdot \log(\#\text{win}) + (1 - \scthreshold)\cdot \log(\#\neg
      \text{win})
    \end{split},
  \end{equation}
  where $\#\text{win}$ and $\#\neg \text{win}$ are the
  number of paths to $\target$ and $\sink$ respectively and equality holds
  for the policy that selects paths uniformly. Furthermore, notice that this policy
  bounds the maximum probability of a trace given the performance:
  \begin{equation}
    \ppthreshold \geq \max\left(\frac{\scthreshold}{\#\text{win}}, \frac{ 1- \scthreshold}{\#\neg \text{win}}\right).
  \end{equation}
  It follows then that the inverse of the Pareto characteristic
  function determines the causal entropy threshold needed to decide the RCI instance
  using RCI, i.e.,
  \begin{equation}
    \randomness = \solfuncp^{-1}(\scthreshold) \cdot h^*.
  \end{equation}
\end{proof}
Thus, one can view ERCI as a conservative extension of RCI to stochastic games.


\subsection{Additional Related Work}\label{sec:related}
 
Synthesis in MDPs with multiple hard and soft constraints (often over indefinite horizons) is a well-studied problem~\cite{DBLP:conf/stacs/ChatterjeeMH06,DBLP:conf/tacas/EtessamiKVY07,DBLP:conf/atva/ForejtKP12,DBLP:journals/fmsd/RandourRS17}.  In this setting, one generates deterministic policies and their convex combinations. Put differently, some degree of randomization is \emph{not an objective}, but rather a consequence. Interestingly, in \cite{DBLP:conf/tacas/DelgrangeKQR20} the optimal policies in \emph{absence} of randomization are investigated. Along similar lines, \cite{DBLP:journals/jcss/BrazdilCFK17} trades average performance for less variance, thereby implicitly trading off the average and the worst-case performance.  
The original results sparked interest in different extension to MDPs and the type of soft constraints, such as continuous MDPs \cite{DBLP:journals/csysl/HaesaertNS21} and continuous-time MDPs~\cite{DBLP:conf/cav/QuatmannJK17},  cost-bounded reachability \cite{DBLP:journals/jar/HartmannsJKQ20}, or mean-payoff properties~\cite{DBLP:journals/corr/abs-1104-3489}. 
The algorithms have also been extended towards stochastic games~\cite{DBLP:conf/mfcs/ChenFKSW13,DBLP:journals/sttt/KwiatkowskaPW18}.
Finally, notions of lexicographic multi-objective synthesis~\cite{DBLP:conf/cav/ChatterjeeKWW20} -- in which one optimizes a secondary criterion among all policies that are optimal with respect to a first criterion bare some resemblance with the algorithm we consider. 
These algorithms have been put in a robotics context in~\cite{DBLP:journals/ijrr/LacerdaFPH19}.
Finding policies that optimize reward objectives is well-studied in the field of reinforcement learning, and has been extended to generate Pareto-fronts for multiple objectives~\cite{DBLP:conf/icml/NatarajanT05,DBLP:conf/adprl/ParisiPSBR14}.

Randomization has been considered in different contexts. Entropy in MDPs is optimized in \cite{DBLP:journals/tac/SavasOCKT20}. 
\textcolor{red}{Marcell, do you have more on (causal) sentropy?}
Beyond Markov models, the (uniform) randomization over languages in finite automata \cite{DBLP:journals/siamcomp/HickeyC83,DBLP:conf/soda/KannanSM95} or over propositional formulae \cite{DBLP:journals/tcs/JerrumVV86,DBLP:journals/iandc/BellareGP00,DBLP:conf/dac/ChakrabortyMV14} has received quite some attention. Neither of those approaches support the notion of soft constraints or the related tradeoffs.

Path-finding has long been considered a multi-objective problem itself~\cite{DBLP:conf/icra/AmigoniG05,DBLP:journals/eswa/NazarahariKD19,DBLP:conf/icml/XuTMRSM20}.
These works differ prominently in two aspects: they do not trade-off randomization and performance, and they do not trade-off declarative and formal constraints with the accompanying formal guarentees, but are more search-based. 

Patrolling POIs and perimeters has received plenty of attention, e.g.,~\cite{DBLP:conf/icra/AgmonKK08,DBLP:conf/icra/AmigoniBG09,DBLP:conf/iros/PortugalPRC14}.  
Closest to our work are  formalisms rooted in game-theory,  such as  \emph{Stackelberg games}~\cite{simaan1973stackelberg,DBLP:conf/atal/ParuchuriPTOK07}. Stackelberg games have been extending to Stackelberg planning~\cite{DBLP:conf/aaai/SpeicherS00K18} in which a tradeoff between the cost for the defender and the attacker can be investigated.
Most related are the zero-sum~\emph{patrolling games} introduced in~\cite{DBLP:journals/ior/AlpernMP11}, which has led to numerous practical solutions~\cite{DBLP:books/daglib/0040483}. Patrolling games are explicitly games between an intruder and a defender, and there is no stochastic environment.  Adding additional objectives makes solving these problems harder~\cite{DBLP:conf/atal/Klaska0R20} and in general, the obtained policies are no longer applicable. To overcome this, a specific set of fixed objectives has been added to these games recently~\cite{DBLP:conf/atal/Klaska0R20}. 
 The large common aspect in all of this work is that optimal strategies do randomize. As in the synthesis work above, this is a consequence of the objectives rather than an objective in itself. 
 In comparison, we provide a general framework and in particular support stochastic environments.
 
%%% Local Variables:
%%% mode: latex
%%% TeX-master: "main"
%%% End:
