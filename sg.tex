
MDP algorithm in hand, we are now ready to provide an algorithm for
stochastic games.
%At a high level, this algorithm works by initially
%planning for $\pTwo$ selecting the action that minimizes randomness
%(with ties broken by performance). This assumption leads to a
%reduction to the MDP-case, where the rationality indexed family,
%$\sched_\rat$, indexes the Pareto Front.  If this assumption is ever
%violated, the resulting state must support more randomness.  The
%rationality, $\rat$, is thus lowered to match the worst case randomness,
%which due to monotonicity of $\solfuncp$, can only increase
%performance. Surprisingly, as we shall later prove, this class of
%policies indexes the Pareto Front for SGs!


\mypara{Environment Policies} We begin with three observations about
the $\pTwo$-policies.  First, for ERCI, we can assume an adversary
that aims to foil $\pOne$ achieving both the performance \emph{and}
randomization requirement.  To do so, it suffices to violate either
performance \emph{or} randomization.  Second, if there is a violating
$\pTwo$-policy, there is a deterministic $\pTwo$-policy that proves
this.  In particular, at every state, $\pTwoSched$ may choose to
violate either constraint via the appropriate action with no incentive
to randomize. Third, fixing an environment policy reduces $\sg$ to a
Markov Decision process, call $\sg[\pTwoSched]$.

\mypara{Algorithmic Idea}
% Before we discuss the algorithm in full detail, we consider two simple
% cases with a single $\pTwo$-state \footnote{To ensure we satisfy our
% our assumption of games being alternating between both players, we can
% insert dummy states.} and multiple $\pOne$-states.  First, we consider
% a SG with an initial $\pTwo$-state, $s_2$, that can transition to two
% independent MDPs (a sub-game with only $\pOne$-states) as illustrated
% in the dashed box in Fig.~\ref{fig:sg:simplest}.  Here, $\pTwo$ can
% force $\pOne$ to play in either of two MDPs that start in states $s_3$
% and $s_4$, respectively.  For both MDPs, we can compute (an
% approximation of) the Pareto front and the corresponding sets of
% achievable points as described in Sec.~\ref{sec:mdps}.  The achievable
% points for state $s_2$ are now given by the intersection of the two
% sets.
% % However, the challenge in this generalization is to select
% % action for $\pOne$-states.
% Now, consider the full SG illustrated in Fig.~\ref{fig:sg:simplest}
% (initial state $s_0$).  To determine $\pOneSched(s_0)$, we must plan
% for whatever action $\pTwo$ selects in $s_2$. Thus, the Pareto front for
% $s_0$ non-trivially depends on the Pareto curves for $s_2$ and
% $s_1$. 
% Algorithmically, we want to avoid computing the (complete)
% Pareto fronts as much as possible. To do so, we restrict our attention
% to policies induced by some rationality $\rat$ (as in the MDP case).
% Later we prove this restriction is w.l.o.g. Conceptually,
% $\rat$ then enables $\pOne$ to associate ``comparable'' points between
% sibling Pareto fronts, and thus reason locally.
Our algorithm operates by picking an optimization direction
$\langle \rat, 1 \rangle$ and \emph{temporarily} assuming that $\pTwo$
selects a deterministic policy, $\sched^\rat_\pTwo$, that
lexicographically minimizes the guaranteed randomness, followed by
performance. On the sub-graph, $\sg[\sched^\rat_\pTwo]$, $\pOne$
employs the corresponding entropy maximizing policy for the MDP,
$\sg[\sched^\rat_\pTwo]$. This partial-policy is extended to a policy
on the rest of the $\sg$ as follows. Whenever $\pTwo$ diverges from
the entropy minimizing policy, it must be possible for $\pOne$ to
trade randomness for performance, i.e. increase $\rat$, while still
ensuring the same randomness against a entropy minimizing
adversary. Thus, a new MDP partial-policy is computed and extended
recursively until the terminal states of the game graph.  We call the increase
in rationality (and the associated computations) \emph{replanning} and
the corresponding family of policies, $\{\sched^\rat_\pOne\}_\rat$,
\emph{entropy matching}.  Finally, we propose approximating the
relevant Pareto front using the same three staged sequence of
rationality coefficients as the MDP case (1) endpoints, (2) doubling,
(3) binary search.

\mypara{Soundness and Completeness}
Importantly, observe that because
fixing a policy yields a verifiable point in $\solutions$, any witness
for realizability we find is trivially sound. For completeness, we can
restrict ourselves to the case in which our algorithm claims the ERCI
instance unrealizable. Surprisingly, the class of policies we consider
suffices, and the algorithm is thus sound and (whenever halting)
complete (proof provided in Sec~\ref{sec:proofs}). That is all
guaranteed points are witnessed by an entropy matching policy!

Further, observe that as a corollary of the entropy matching family
being complete, it must be the case that $\solfuncp(\rndp_\rat)$
inherits continuity and (strict) monotonicity from the MDP
case. Namely, at each $\pTwo$ state, the achievable points
$\solutions$ are necessarily the intersection of the achievable points
of the sub-graphs. By induction, (with the MDP base case), we obtain
continuity and strict monotonicity.

\mypara{Memoizing Pareto Fronts} To perform the above computations
efficiently, we adopt a geometric perspective. Namely, observe that
each node of $\sg$ indexes a sub-graph, which has a corresponding
Pareto front for trading performance for randomness. Further, note
that the Pareto front at an $\pTwo$ node is the intersection of the
Pareto fronts of its child nodes. Entropy matching thus corresponds to
``jumping'' between Pareto fronts and adjusting the optimization
direction by increasing the rationality.  \todo[inline]{Visualize this
  in terms of intersections and different directions. Attach to game
  graph and annotate up arrow for dynamic programming and down error
  for MDP subroutine} Thus, by traversing the graph from the terminal states to
the initial state, approximating Pareto Fronts along the way, one can memoize
how to trade performance for randomness at any given node. This
preprocessing enables determining the minimum entropy response for any
optimization direction and quickly replanning via a convex combination
of Pareto optimal policies.


%Finally, these responses can be computed via a standard dynamic programming (topologically from terminal states to the initial state) operation. 

%First and foremost, observe that to render an ERCI instance
%un-achievable, it suffices for $\pTwo$ to violate
%either the performance threshold \emph{or} the randomness threshold.  Next,
%notice that because $\pOneSched$ is a priori fixed, $\pTwoSched$ can
%be seen as repeatedly selecting between a convex combination of
%$\rndp$ (and $\scp$) for the corresponding sub-graph. As the maximum of a
%convex combination is always achievable on the boundaries, we can
%w.l.o.g. assume that $\pTwoSched$ is \emph{deterministic}.  Similarly,
%notice that given a fixed $\pOneSched$, the worst-case $\pTwoSched$
%response can be computed via dynamic programming in topological order
%from the leafs to the root of $\sg$.
\begin{figure}[h]
\centering
\scalebox{0.8}{
\begin{tikzpicture}
    \node[sstate, initial, initial text=,initial where=left] (a1) {$s_0$}; 
    
    \node[right=of a1,color=blue] (a1r) {$\lambda$ }; 
    \draw[dotted,blue, thick] (a1) -- (a1r);
    
	\node[sstate,below=0.6cm of a1,xshift=8em] (a2) {$s_1$};
	
	\node[astate,below=0.6cm of a1,xshift=-8em] (a0) {$s_2$};
	\node[sstate,below=0.6cm of a0,xshift=-5em] (s1) {$s_3$};
	\node[sstate,below=0.6cm  of a0,xshift=7em] (s2) {$s_4$};
	
	
	
	\node[below=0.1cm of s1, inner sep=0.3pt] (x1) {};
	
	\node[below=0.1cm of s2, inner sep=0.3pt] (x2) {};
		\node[below=0.1cm of a2, inner sep=0.3pt] (x3) {};
	
	\draw[->] (a0) -- node[actnode] (x) {}
					  node[pos=0.4,elab,right,xshift=2mm] {$a$}  (s1);
	\draw[->] (a0) -- node[actnode] {}
					  node[pos=0.4,elab,left,xshift=-2mm] {$b$}  (s2);
					  
	\draw[->] (a1) -- node[actnode] {}
					  node[pos=0.4,elab,right,xshift=2mm] {$a$}  (a0);
	\draw[->] (a1) -- node[actnode] {}
					  node[pos=0.4,elab,left,xshift=-2mm] {$b$}  (a2);
					  
	\draw[->, red] (a0) -- (x);
	
	\node[right=0.4cm of s1,color=blue] (s3r) {$\lambda$ }; 
    \draw[dotted,blue, thick] (s1) -- (s3r);
    \node[right=0.4cm of s2,color=blue] (s4r) {$\lambda' \geq \lambda $}; 
    \draw[dotted,blue, thick] (s2) -- (s4r);
    \node[right=0.4cm of a2,color=blue] (a2r) {$\lambda$ }; 
    \draw[dotted,blue, thick] (a2) -- (a2r);
    \node[left=0.4cm of s1,color=red] (s3r) {$\lambda \mapsto h_3$ }; 
    \draw[dashed,red, thick] (s1) -- (s3r);
    \node[left=0.4cm of s2,color=red] (s4r) {$\lambda \mapsto h_4 \geq h_3$}; 
    \draw[dashed,red, thick] (s2) -- (s4r);
    \node[left=0.4cm of a2,color=red] (a2r) {$\lambda \mapsto h_1$ }; 
    \draw[dashed,red, thick] (a2) -- (a2r);
    \node[left=0.4cm of a0,color=red] (s2r) {$\lambda \mapsto h_3$ }; 
    \draw[dashed,red, thick] (a0) -- (s2r);
    			  
				  
	
	\draw[dashed] (x1) -- +(1,-1) -- +(-1,-1) coordinate  (fa) -- (x1);
	\draw[dashed] (x2) -- +(1,-1) coordinate (fb) -- +(-1,-1)  -- (x2);
	
	\draw[dashed] (x3) -- +(1,-1) -- +(-1,-1)  -- (x3);
	
	\node[below=3mm of x1] {{\tiny MDP}};
	\node[below=3mm of x2] {{\tiny MDP}};
	\node[below=3mm of x3] {{\tiny MDP}};
	
%	\node[fit=(a0)(x1)(x2)(fa)(fb), inner sep = 6pt, dotted,draw] {};
	
	
\end{tikzpicture}

%%% Local Variables:
%%% mode: latex
%%% TeX-master: "main"
%%% End:

}
\caption{Simple SG to illustrate intuition}
\label{fig:sg:simplest}
\end{figure}

% \mypara{Minimum Entropy Sub-graphs} Next, we formalize the
% environment's entropy minimizing counter-policy.  First, assume $\rat$
% is a priori fixed. Then, note that the minimum entropy
% counter-strategy to $\rat$, call $\sched_\pTwo^\rat$, and the
% corresponding $\pOne$ partial\footnotemark policy,
% $\sched_\pOne^\rat$, can be computed from the sinks to root as
% follows.
% \begin{enumerate}
% \item \textbf{Sink}:
%   No actions, thus $\sched^\rat = \langle \sched_\pOne^\rat, \sched_\pTwo^\rat\rangle$
%   is trivial.
% \item \textbf{$\pTwo$-node}: Suppose $\sched^\rat$ is known for the
%   sub-graphs accessed by each action. By assumption,
%   $\sched_\pTwo^\rat$ will pick the entropy minimizing action (with
%   ties broken by performance).
% \item \textbf{$\pOne$-node}: suppose $\sched_\pTwo^\rat$ is fixed,
%   reducing the SG to a MDP, with corresponding policy
%   $\sched_\pOne^\rat$. Importantly, because of the uniqueness of
%   $\sched_\pOne^\rat$ on MDPs, $\pOne$'s policy remains unchanged on
%   all sub-graphs accessed by its actions. Thus $\sched_\pTwo^\rat$
%   remains unchanged.


% \end{enumerate}
%   \footnotetext{These policies are no longer unique because some
%   parts of the SG may be unreachable under $\sched_\pTwo^\rat$}
% %
%On the other hand, suppose $\pTwoSched$ was known, reducing the SG to
%a MDP. As discussed in the previous section, for the MDP case, it
%suffices to consider rationality indexed policies. Let use denote the
%resulting MDP and maximum causal entropy family of (partial) policies
%as $\sg[\pTwoSched]$ and $\pi^\pOne_\rat[\pTwoSched]$ resp. Now
%suppose $\rat$ is a priori fixed. Again, via a topological ordered
%evaluation of states from the leaves to the root, one can compute the
%$\pTwo$ policy, call $\sched_\rat^\pTwo$, that minimizes the maximum
%causal entropy policy family, $\pi^\pOne_\rat[~.~]$. By
%We shall refer to resulting (partial) schedule as: $\pi_\rat \eqdef \langle
%\pi^\pOne_\rat[\sched_\rat^\pTwo] , \sched_\rat^\pTwo \rangle$.

% \mypara{Replanning}
% Of course, even if $\rat$ is fixed, $\pTwoSched$ need not be
% $\sched^\pTwo_\rat$. Nevertheless,  by selecting the worst
% entropy MDP, $\sched_\rat$ establishes an achievable randomness for the
% sub-game rooted at each state, call $\rndp_\rat(s)$. Now suppose,
% $\pTwo$ deviates from $\sched_\rat^\pTwo$ at state $s$. Note that, so
% long as $\pOneSched$  yield randomness less than
% $\rndp_\rat(s)$ at the new successor-state, the worst-case randomness will not decrease.
% This begs the question ``what maximum performance can $\pOne$ guarantee at $s' \in
% \Succ(s)$ given randomness $\rndp_\rat(s)$?'' This question can be investigated on a sub-game, and eventually, must be answered on an MDP. 
% \begin{mdframed}
%   The key observation is that each state $s'$ is the root of a sub
%   game, $\sg[s]$, with a corresponding Pareto Front,
%   $\pareto{\solutions}[s]$.
% \end{mdframed}
% In particular, given access the characteristic functions,
% $f_\solutions^{s'}$, for each $s' \in \Succ(s)$, one can compute:
% \begin{equation}\label{eq:performance_lookup}
%   \epsilon_\rat(s) = \min_{s'} f_\solutions^{s'}(\delta_\rat(s)).
% \end{equation}
% Thus, via dynamic programming, one can define $\epsilon_\rat(\iota)$.
% Moreover, assuming that this entropy matching family of policies
% indexes $\pareto{\solutions}$ (proved in~Sec.~\ref{sec:proofs}), one can handle
% deviations from $\sigma^\pTwo_\rat$ by ``replanning''. Namely, one
% extends $\pi_\rat$ at $s'$ by computing a new rationality coefficient,
% $\rat'$, such that:
% \begin{equation}
%   \delta_{\rat'}(s') = \delta_{\rat}(s)
% \end{equation}
% Due to the monotonicity $\rat \leq \rat'$, and thus $\epsilon_\rat \leq \epsilon_{\rat'}$, 
% Therefore, ignoring the feasibility of computing the exact Pareto
% Front, we obtain a synthesis algorithm for SGs!
\todo[inline]{Work out for running example?}
\mypara{Approximate Pareto Fronts}
Of course, by varying $\rat$, one can only construct approximate
Pareto fronts $\hat{\pareto{}} \subseteq \pareto{\solutions}$, where we denote the downward closure of $\hat{\pareto{}}$ as $\hat{\solutions} \subseteq \solutions$.
% let $\hat{f}_\solutions^{s'}$ denote the characterising function.
We propose the following high-level algorithm to adapt the above
algorithm to the case where each Pareto front approximation introduces at
most $\kappa$ error along the performance axis.
\begin{mdframed}
\begin{enumerate}
\item Let $\tau$ denote the length of the longest path in $\sg$.
\item Let $0 < \kappa < 1$ be some arbitrary initial tolerance.
\item Recursively compute $\kappa$-close Pareto fronts for each successor state using replanning.
\item If the any minimum entropy action cannot be determined or $\scthreshold$ is within $\kappa\cdot \tau$ distance to (but outside of) $\hat{\pareto{}}$,
  halve $\kappa$ and repeat.
\item Otherwise, perform entropy matching algorithm (with initial
  entropy $\randomness$) using these Pareto Fronts and return the
  resulting policy (if on exists).
\end{enumerate}  
\end{mdframed}
The soundness of this algorithm relies on the following critical
facts: (1) Given sufficient resolution, the minimum entropy
$\pTwo$-actions can be determined and the resulting entropy. (2) The
resulting entropy is depends solely on
the resulting sub-graph (and is independent of the current Pareto
approximation). (3) Thus, when querying points on
$\pareto{\solutions}$, error can only accumulate for $\scp$. (4) Next,
observe that $\scp$ is computed using convex combinations of entropy
matched points on Pareto approximations. (5) Convex combinations of an error interval cannot
increase the error, i.e.,
\begin{equation}
  q\cdot[x, x + \kappa] + \bar{q}\cdot[y, y + \kappa] = [z, z + \kappa],
\end{equation}
where $z = q\cdot x + \bar{q}\cdot y$.
Thus, so long as $\kappa\cdot\tau$ is enough resolution to answer $\scp_\rat <
\scthreshold$, one obtains a semi-decision procedure as in the MDP
case.

\mypara{Termination and Run Time} First, as in the MDP case, the
algorithm terminates almost surely, with the exception occurring only
for a subset of the Pareto Front.  Below, we give an output-sensitive
analysis of the run time (assuming it does halt).  If $\kappa^*$
tolerance is required to terminate, then the $\kappa$ search
introduces $\mathcal{O}(\log(\nicefrac{1}{\kappa^*}))$
iterations. Next, observe that each node processes a given rationality
coefficient at most once. Further, looking up which pair of rationalities
are need to upper and lower bound the performance for a given randomness
can be in logarithmic time via binary search on rationality coefficients.
As the corresponding bounds and convex combinations can be computed in
constant time, this means this algorithm runs in time:
\begin{equation}
  \label{eq:sg-runtime}
  \mathcal{O}(\log(\nicefrac{1}{\kappa^*})\cdot N_\rat\cdot \log(N_\rat)\cdot |\sg|),
\end{equation}
where, $N_\rat$ is the number of unique rationality coefficients
processed.  If, as in the MDP case, one assumes a maximum rationality
coefficient $\rat^*$ and a minimum rationality resolution $\Delta$,
one obtains:
\begin{equation}
  \mathcal{O}(\underbrace{\log(\nicefrac{1}{\kappa^*})}_{\kappa \text{ search}}\cdot \underbrace{\nicefrac{\rat^*}{\Delta}\cdot \overbrace{\log(\nicefrac{\rat^*}{\Delta})}^{\text{Replanning}}\cdot |\sg|}_{\text{Evaluate } \rat}).
\end{equation}
The above however is very conservative and empirically we observe
$N_\rat$ bounded far away from $\nicefrac{\rat^*}{\Delta}$.

%In practice, this algorithm can be significantly improved by adaptive
%tolerances, lazily computing the Pareto Fronts, and only computing
%Pareto Fronts for $\pTwo$ states. Nevertheless,
%already this na\"ive algorithm gives a sense of the run-time
%bottlenecks. Namely, if $\kappa^*$ tolerance is required to terminate,
%then the $\kappa$ search introduces $O(\log(\nicefrac{1}{\kappa^*}))$
%iterations. Furthermore, by computing $O(|\sg|)$ Pareto fronts, from the
%leaves, one ensures that the complexity grows linearly with the graph
%size - although the multiplicative constant depends on the number of
%rationality coefficients explored per Pareto front. We found that most
%of the rationality coefficients explored were shared, which in
%practice seems to amortize the cost per state. Finally, as in the
%MDP-case, the algorithm halts if $\langle \scthreshold, \randomness \rangle$ is
%bounded away from the root Pareto Front. As the Pareto Front has
%measure 0, we argue that this is merely a technical concern, as a
%small perturbation to the ERCI instance (i.e. a Smoothed
%Analysis~\cite{SmoothedAnalysis}) on $\sg$ admits decidability.


%%% Local Variables:
%%% mode: latex
%%% TeX-master: "main"
%%% End:
