\section{Proofs}\label{sec:proofs}
\mypara{Convexity of ERCI solution set}
\begin{proof}[Proof Sketch Prop~\ref{prop:convex}]
  Recall that a set is convex, if it is closed under
  convex-combinations\footnotemark. Consider two points
  $\langle \scp, \rndp \rangle, \langle \scp', \rndp' \rangle \in
  \solutions$ achieved by $\pOneSched$ and $\pOneSchedPrime$
  respectively. Consider the new policy, $\pi$, defined by employing
  $\pOneSched$ with probability $q$ and $\pOneSchedPrime$ with
  probability $\bar{q} \eqdef 1 - q$.  Because each policy
  \emph{guarantees} its corresponding performance, this new policy as
  performance at least $q\cdot \scp + \bar{q}\cdot \scp'$.  Similarly,
  by viewing $\pi$ as a random variable and applying chain rule
  yields,
  \begin{equation}
    \begin{split}
      H_\tau(\sigma)
      \geq &~q \cdot H( \rv{A}^{\pOne}_{1:\tau'} \mid\mid \rv{S}_{1:\tau} \mid \pi=\pOneSched)~+\\
      &~\bar{q}  \cdot H( \rv{A}^{\pOne}_{1:\tau'} \mid\mid \rv{S}_{1:\tau} \mid \pi=\pOneSchedPrime)\\
      =&~q\cdot\rndp + \bar{q}\cdot \rndp'.
    \end{split}
  \end{equation}
  Thus, any convex combination of guaranteed points is guaranteed by
  a convex combination of the corresponding ego policies.
\end{proof}


\mypara{Completeness of Entropy Matching}
Below, we shall
assume that our algorithm claims the ERCI is un-achievable, but there
exists another policy not in $\{\sched^{\pOne}_\rat\}_\rat$, that
achieves $\langle \scthreshold, \randomness \rangle$. Before our proof,
we introduce the following lemma and corollary.

\begin{lemma}
  Let $\{\sched_\rat\}_\rat$ be the family of entropy matching policies
  with corresponding $x_\rat = \langle \scp_\rat, \rndp_\rat\rangle$,
  \begin{equation}
  \rat < \rat' \implies (\scp_{\rat} \leq \scp_{\rat'}) \wedge (\rndp_\rat \geq \rndp_{\rat'}).
  \end{equation}
\end{lemma}
\begin{corollary}\label{cor:termination}
  Assume $\rndp_0 \geq \randomness$ and
  let $\rat^*$ be the smallest rationality coefficient such that $\randomness = \rndp_\rat$.
  The following two statements are equivalent.
  \begin{equation}
       \exists \sched_\pOne^* \in \{\sched^{\pOne}_\rat\}_\rat~.~  \langle \scthreshold, \randomness \rangle \preceq x_{\sched_{\pOne}^*}.
  \end{equation}
  \begin{equation}
    \langle \scthreshold, \randomness \rangle \preceq x_{\rat^*}.
  \end{equation}
\end{corollary}
\begin{proof}[Proof Sketch of SG Completeness]
  First, observe that on games with only sink nodes, completeness
  follows directly.
  Now suppose the entropy matching family is
  complete on all sub-graphs of $\sg$ and for the sake of
  contradiction that, assume that,
  \begin{align}
    &\forall \sched_\pOne \in \{\sched^{\pOne}_\rat\}_\rat~.~ x_{\sched_{\pOne}} \prec \langle \scthreshold, \randomness \rangle\label{eq:reject}\\
    &\exists \sched_\pOne^* \notin \{\sched^{\pOne}_\rat\}_\rat~.~  \langle \scthreshold, \randomness \rangle \preceq x_{\sched_{\pOne}^*}\label{eq:incomplete}.
  \end{align}
  Next, as in Corollary~\ref{cor:termination}, let $\rat^*$ be the smallest
  rationality such that $\randomness = \rndp_\rat$. Such a rationality must
  exist, otherwise the ERCI instance is trivially infeasible since the
  entropy matching family contains the minimizers and maximizers of entropy.


  Next, let
  $\sched_\pTwo^*$ denote the min-randomness $\pTwo$-policy given
  $\sched_{\rat^*}$. By~\eqref{eq:reject}, this policy must have
  insufficient performance.
  % On the MDP, $\sg[\sched_\pTwo^*]$, we know that
  % that the soft-bellman backup family is complete, and thus there must
  % exist some $\rat$ such that the partial policy $\pi_\rat$ matches
  % the worst-case entropy of $\sched_\pOne^*$, i.e.,
  % $\rndp_\rat = \rndp_{\sched_\pOne^*}$.
  Furthermore, observe that  (1) $\sched_\rat$ indexes the pareto front for MDPs and (2) $\rndp_{\rat^*} = \randomness$. Therefore, it must be
  the case that:
  \begin{equation}
    \scp_{\langle \sched_{\pOne}^*, \sched_{\pTwo}^* \rangle} \leq \scp_{\langle \sched_{\rat^*}, \sched_{\pTwo}^* \rangle} < \scthreshold.
  \end{equation}
  This however leads to a contradiction to \eqref{eq:incomplete}.  Thus,
  entropy matching must be complete.
\end{proof}

\mypara{ERCI and RCI coincide}
\begin{proof}[Proof Sketch of Prop~\ref{prop:conservative}]

  First, observe that as in ERCI, for RCI, one can w.o.l.g. assume
  $\pTwoSched$ is deterministic~\cite{DBLP:conf/cav/FremontS18}.
  Thus, given some worst-case adversary (for randomness and
  performance) all randomization is due to the behavior of
  $\pOne$. Assuming a non-trivial solution set (and thus
  $\scopt \neq 0$), this implies that $\pOne$ can \emph{first} decide
  whether to meet the soft constraint and \emph{then} choose actions
  weighted by the number of guaranteed ways to reach $\target$ in the
  corresponding subtree. Now observe when planning against the
  min-causal entropy adversary, $\sched_*^\pTwo$, the deterministic SG
  reduces to a deterministic MDP. In deterministic MDPs, the action
  sequence uniquely determines the state sequence, and thus the causal
  entropy reduces to the entropy of the actions and states, i.e., the
  paths, $\xi$.  Thus, treating $\xi$ as a random variable and
  case splitting on winning yields:
  \begin{equation}\label{eq:cond_win}
    \begin{split}
      h_{\langle \sched_\pOne, \sched^*_\pTwo\rangle} &= \scthreshold\cdot H(\xi \mid \text{win}) + (1 - \scthreshold)\cdot H(\xi \mid \neg \text{win})\\
      &\leq \scthreshold\cdot \log(\#\text{win}) + (1 - \scthreshold)\cdot \log(\#\neg
      \text{win})
    \end{split},
  \end{equation}
  where $\#\text{win}$ and $\#\neg \text{win}$ are the
  number of paths to $\target$ and $\sink$ respectively and equality holds
  for the policy that selects paths uniformly. Furthermore, notice that this policy
  bounds the maximum probability of a trace given the performance:
  \begin{equation}
    \ppthreshold \geq \max\left(\frac{\scthreshold}{\#\text{win}}, \frac{ 1- \scthreshold}{\#\neg \text{win}}\right).
  \end{equation}
  It follows then that the inverse of the Pareto characteristic
  function determines the causal entropy threshold needed to decide the RCI instance
  using RCI, i.e.,
  \begin{equation}
    \begin{split}
      \randomness &= \rndmin + \rndopt\cdot (\delta - \rndmin)\\
      &= \rndmin + \rndopt \cdot \Big (\solfuncp^{-1}\left(\frac{\scthreshold - \scmin}{\scopt}\right) - \rndmin\Big)
    \end{split}.
  \end{equation}
\end{proof}

Thus, one can view ERCI as a conservative extension of RCI to stochastic games.

%%% Local Variables:
%%% mode: latex
%%% TeX-master: "main"
%%% End:
