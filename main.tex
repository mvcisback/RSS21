\documentclass[runningheads]{llncs}
\usepackage{microtype}
\usepackage{amsmath,amssymb}
\usepackage{paralist}
\usepackage{mdframed}
\usepackage{xcolor}

\newcommand{\NN}{\mathbb{N}}
\newcommand{\mc}{\mathcal{D}}
\newcommand{\sg}{\mathcal{G}}
\newcommand{\eventually}[1]{\lozenge^{\leq #1}}
\newcommand{\sched}{\sigma}
\newcommand{\Sched}{\Sigma}
\newcommand{\pol}{\sched}
\newcommand{\Distr}{\ensuremath{\textsl{Distr}}}
\newcommand{\act}{\alpha}
\newcommand{\Act}{A}
\newcommand{\last}[1]{{#1}_\downarrow}
\newcommand{\Paths}[2][]{\Pi^{#2}_{#1}}
\newcommand{\POnePaths}[2][]{\Pi^{\downarrow_{1}#2}_{#1}}
\newcommand{\PTwoPaths}[2][]{\Pi^{\downarrow_{2}#2}_{#1}}
\newcommand{\PiPaths}[2][]{\Pi^{\downarrow_{i}#2}_{#1}}
\newcommand{\unrolled}[2]{\textsf{Tree}(#1,#2)}
\newcommand{\induced}[2]{#1[#2]}


\setlength\marginparwidth{110pt}
\newcommand{\colorpar}[3]{\colorbox{#1}{\parbox{#2}{#3}}}
\newcommand{\marginremark}[3]{\marginpar{\colorpar{#2}{\linewidth}{\color{#1}#3}}}
\newcommand{\commentside}[2]{\marginpar{\color{#1}\tiny#2}}
\newcommand{\TODO}[1]{\commentside{teal}{\textsc{Todo:} #1}}
\newcommand{\REMARK}[1]{\commentside{teal}{\textsc{Remark:} #1}}\newcommand{\sj}[1]{\marginremark{black}{red!10!white}{\scriptsize{[SJ]~ #1}}}
\newcommand{\mvc}[1]{\marginremark{black}{gray!10!white}{\scriptsize{[MVC]~ #1}}}

\title{On Control Improvisation for Stochastic Games}
\author{Marcell Vazquez-Chanlatte \and Sebastian Junges \and Sanjit A.\ Seshia}
\institute{University of California, Berkeley, CA, USA}

\begin{document}
\maketitle\sj{Daniel?}
\begin{abstract}
	Efficacious controller synthesis is a key ingredient in the design and analysis of complex systems. We study the design of controllers that have a high entropy, that is, whose behavior or nature is surprising. The synthesis of such controllers is key in domains like testing and security. 
	In particular, our paper studies control improvisation and compares them with randomly sampling adequate policies. The only difference in obtained policies is in their notion of entropy, but the problems are significantly different.  We illustrate and contrast their merits and limitations. Furthermore, we provide algorithms that solve both control improvisation problems. Prominently, we solve the control improvisation problem for Markov decision processes by relating it to recent results from inference from demonstrations, and then extend this approach to stochastic games. We present a prototypical implementation that efficiently solves controller synthesis problems from the security and testing domain. 
\end{abstract}
\section{Introduction}
% Declarative Constraints ar neat idea.
The use of declarative specifications, e.g. in the form of temporal logic formulas, has become a popular way to construct high-level robot controllers~\cite{DBLP:conf/iros/HorowitzWM14, DBLP:conf/rss/WongEK14, DBLP:conf/iros/HeLKV17, DBLP:conf/icra/FuATP16, DBLP:conf/icra/HeWKV19, DBLP:journals/arobots/MoarrefK20, DBLP:conf/icra/KantarosM0P20}.
% Synthesis closes the gap.
Given a user provided specification, \emph{synthesis} algorithms aim
to automatically create a control policy that ensures that the
specification is met, or explain why such a policy does not
exist. Together, synthesis and declarative specifications facilitate
quickly and intuitively solving a wide variety of control tasks.  For
example, consider a delivery drone operating in a workspace. One may
specify the drone should ``within 10 minutes, visit four locations (in any
order) \emph{and} avoid crashing.''. A synthesis tool may then create a
finite state controller which guarantees this specification is met,
under a particular world model.
% Declarative Synthesis need not produce variety.  
Importantly, while there may be many controllers that conform to the
provided specification, many synthesis algorithms provide a
single, often deterministic, policy.  For instance, in our drone
example, a synthesized controller may generate only a single path
through the workspace.

% On the importance of being varied.
In some settings, such policies however are undesirable.  First, in
many tasks, the predictability (or bias) of the policy may be a
liability.  Example include
patrolling~\cite{DBLP:journals/ior/AlpernMP11}, behavior prediction
and inference~\cite{DBLP:conf/cav/Vazquez-Chanlatte20}, and creating
controller harnesses for fuzz testing (see motivating
example). Second, synthesis algorithms work on \emph{idealized}
models, and thus any policy that over commits to any given model quirk
may in practice yield poor performance. In such settings,
randomization is known to make policies more robust against worst-case
deviations~\cite{mceThesis, maxEntAnswer}. Unfortunately, traditional
synthesis problems result policies that need not (and typically do
not) exhibit randomization.

% Propose CI and highlight new features.
To address these potential deficits, we advocate for the adoption of
the recently proposed control
improvisation~\cite{DBLP:conf/cav/FremontS18,DBLP:conf/fsttcs/FremontDSW15}
framework, in which one specifies a controller with three types of
declarative constraints. (1) \emph{Hard constraints} that, as in the
classical setting, must be satisfied, (2) \emph{soft constraints} that
on most executions should hold, and (3) \emph{randomization
constraints} that ensure that a synthesized policy does not overly
commit to a particular action or behavior. 
The key challenge is that randomization and performance in the form of soft constraints form a natural trade off.

Unfortunately, control
improvisation has so far been limited to deterministic domains where
uncertainty is resolved adversarially. This assumption is often too
restrictive and leads (together with the soft/hard constraints) to
conservative policies or common situations in which the synthesis
algorithm cannot be employed at all. To overcome this weakness, we
develop a theory of control improvisation in stochastic games which
admit arbitrary \emph{combinations} of adversarial and probabilistic
uncertainty, including unknown or imprecise transition
probabilities. as the policy is no longer the only
source of randomization, this extension requires a different
view on randomization constraints.

Technically, we formulate our problem on \emph{simple stochastic
games}~\cite{DBLP:conf/dimacs/Condon90}, an extension of Markov decision processes that divides states
between controllable states and uncontrollable (or adversarially
controlled) states. \emph{Soft constraints} are finite horizon
temporal properties with a threshold on the worst-case probability of
that the property holding by the end of the episode. \emph{Hard
constraints} are soft constraints satisfied with probability 1. In
contrast to other work on control improvisation, we adopt the notion
of causal entropy as natural means to formalize \emph{randomness
constraints}.  Causal entropy is a prominent notion in directed
information theory \sj{Add citation?} that strongly correlates with robustness in the
(inverse) reinforcement learning setting~\cite{mceThesis,
maxEntAnswer}. We refer to this variant of control improvisation as
Entropic Reactive Control Improvisation (ERCI) and show that ERCI
conservatively extends reactive control improvisation~\cite{DBLP:conf/cav/FremontS18} to stochastic
games. More precisely, while we focus on stochastic games, entropy can
be used in the non-stochastic setting and yields results analogous to
the reactive control improvisation. ERCI also conservatively extends  classical policy synthesis in stochastic games, i.e. synthesis in absence of randomness constraints as, e.g., implemented in PRISM-games~\cite{DBLP:journals/sttt/KwiatkowskaPW18}.


%We argue that soft constraints can naturally be considered as an
%optimization objective which one can trade-off for more randomization.
%Indeed, our method strongly relies on the computation of a
%Pareto-front that explores the trade-off between randomization and
%optimizing the soft constraint using the notion of rationality. This
%means that rather than asking the user to fix rather arbitrary
%threshold values for both types of constraints, we may visualize the
%trade-off between these two entities.
%
\mypara{Contributions}
In summary, this paper contributes ERCI, an algorithmic way to trade 
performance and randomization in stochastic games. As we motivate in the example below, the support for stochastic games that combine both
adversarial and probabilistic behavior in an environment allows for
modeling flexibility, admitting applicability to new domains. To
support this extension, the paper proposes and shows the benefits
formulating randomization constraints with causal entropy.  Finally,
this paper contributes the necessary machinery as well as a
prototypical implementation.

\mypara{Overview} This paper is structured as follows. We begin
with a motivating example (Sec.~\ref{sec:motivating}). Then we
provide preliminaries and formalize the ERCI problem statement in
Sec.~\ref{sec:problem}. Next, in Sec.~\ref{sec:convex}, we cast ERCI
as a multi-objective optimization problem and study properties of the
solution set. With this technical machinery developed,
Sec.~\ref{sec:mdps} re-frames existing literature on maximum causal
entropy inference and control to derive an algorithm for Markov
Decision Processes.  Then in Sec.~\ref{sec:sgs}, we provide an
algorithm for the general case of stochastic games. Finally, we
conclude with an empirical evaluation (Sec.~\ref{sec:empirical}) and a
comparison with other control improvisation formulations and other
related work (in Sec.~\ref{sec:related}).



%%% Local Variables:
%%% mode: latex
%%% TeX-master: "main"
%%% End:

\section{Problem Statement}
\subsection{Stochastic Games}
An (alternating, 2.5-player) \emph{stochastic game} (SG) is a tuple $\sg = \langle S, \iota, \Act, P \rangle$. A finite state $S = S_1 \cup S_2 \cup S_E$ is partitioned into a set $S_1$ of player-1 states, a set $S_2$ of player-2 states, and a set $S_E$ of environment states. $\iota \in S_1$ is the initial state, $\Act = \Act_1 \cup \Act_2$ is a finite set of actions, and $P\colon S \times \Act \rightarrow \Distr(S)$ is defined by a set of three transition functions: $P_1\colon S_1 \times \Act_1 \rightarrow S_2$, $P_2\colon S_2 \times \Act_2 \rightarrow S_E$, $P_E\colon S_E \rightarrow \Distr(S_1)$.
If $\Act_2$ is a singleton set, then $\sg$ is an \emph{Markov decision process}.
If both $\Act_1$ and $\Act_2$ are singleton sets, then $\sg$ is a \emph{Markov chain}. If $P_E(s)$ is a Dirac distribution for every $s \in S_E$, then, $\sg$ is called \emph{deterministic}.

In this paper, it is helpful to consider $P_E$ as being defined using an auxiliary notion of environment actions $\Act_E$, a deterministic environment transition relation $P_{\hat{E}}\colon S_E \times A_E \rightarrow S_1$ and (memoryless, randomized) environment-scheduler $S_E \rightarrow \Distr(A_E)$.
\sj{I want to put this text where we use this for the first time.}

A finite path $\pi = s_0 \xrightarrow{a} s_1 \xrightarrow s_2$

\paragraph{Policies.} 
As standard, before we can define probabilities, all nondeterminism needs to be resolved. We do this with the notion of a policy. 

\sj{add unrolling}


\paragraph{Properties.}
We consider finite horizon reachability properties.

\paragraph{Entropy}


Define entropy on a random variable.\sj{Do}

Define entropy on a Markov chain\sj{Do}



\subsection{Control Improvisation and random policies}

\begin{mdframed}
Given a SG $\sg$ with target-states $T$ and $G$ and a horizon $h$, does there exists a policy $\sched_1 \in \Sched_1$  such that for every policy $\sched_2 \in \Sched_2$ and with $\sched = \langle \sched_1, \sched_2 \rangle$ it holds that 
\begin{compactenum}
	\item $\Pr^\sg_{\sched}(\eventually{h} T) \geq 1$
	\item $\Pr^\sg_{\sched}(\eventually{h} G) \geq \lambda$
	\item $H(\sg[\sched]) \geq \kappa$
\end{compactenum}
\end{mdframed}
Rather than fixing $\kappa$ a priori, we are often interested in limiting the \emph{regret}: The last point then becomes:
$H(\sg[\sched]) \geq (1-\delta) \cdot H(\sg[\sched^{*}])$, where $\sched^{*}$ .... \sj{I am not sure how to define this concisely.} 


Before we continue, we want to establish that Control improvisation problem is a conservative extension of deterministic case as investigated in~\cite{}.
\begin{lemma}
	
\end{lemma}


\begin{mdframed}

\begin{compactenum}
	\item $\Pr^\sg_{\langle \sched_1,\sched_2 \rangle}(\eventually{h} T) \geq 1$
	\item $\Pr^\sg_{\langle \sched_1,\sched_2 \rangle}(\eventually{h} G) \geq \lambda$ 
\end{compactenum}
\end{mdframed}
\sj{Define randomly selected policy}



\section{Maximum Entropy Traces and Policy Improvisation}
In this section, we discuss the nature of the randomness as imposed by the CI problem, and relate this to improvising policies, i.e., to randomly sampling policies that satisfy the hard and soft constraint.

\begin{mdframed}

\begin{compactenum}
	\item $\Pr^\sg_{\langle \sched_1,\sched_2 \rangle}(\eventually{h} T) \geq 1$
	\item $\Pr^\sg_{\langle \sched_1,\sched_2 \rangle}(\eventually{h} G) \geq \lambda$ 
\end{compactenum}
\end{mdframed}
\sj{Define randomly selected policy}




\section{Solving the Control Improvisation Problem for MDPs}
We present an algorithm for the control improvisation problem for MDPs. The key insight is to reuse ideas from policy inference from specifications~\cite{}. In the next section, we extend this idea to SGs.

\section{Solving the Control Improvisation Problem for SGs}

\section{Empirical Evaluation}

\end{document}
